\documentclass[presentation]{beamer}
\usepackage{../oop-slides}


\title[OOP14: Reflection]{14 \\ Reflection \\ e informazioni run-time sui tipi, annotazioni e testing}

\begin{document}

\newcommand{\ecl}{\idepath{p14reflection}}

\frame[label=coverpage]{\titlepage}

\fr{Outline}{
  \bl{Goal della lezione}{\iz{
  \item Illustrare il concetto di run-time type information
  \item Mostrare le principali funzionalità della Reflection
  \item Descrivere il meccanismo delle annotazioni
  }}
  \bl{Argomenti}{\iz{
  \item Oggetti \cil{Class} e loro uso
  \item Reflection API
  \item Annotazioni di Java
  \item Testing con \texttt{JUnit}
  }}
}

%\fr{TODO}{
%  \bl{Da aggiungere}{\iz{
%  \item Asserzioni
%  \item Come ingegnerizzare i test?
%  \item Esempio di app?
%  }}
%}

\section{Classi, caricamento e JVM}

\frs{5}{Il classfile}{
  \bl{Classi e JVM}{\iz{
    \item Ogni classe Java (interfaccia, enumerazione, inner o outer) produce un file \texttt{.class} ad opera del compilatore
    \item Nel caso di classi ``inner'' il nome di tale \texttt{.class} è \texttt{<outer>\$<inner>}, nel caso di anonima è \texttt{<outer>\$<numero>}
    \item Tale \texttt{.class} è disponibile nel folder di uscita e innestato a seconda del suo package
    \item Il contenuto informativo del \texttt{.class} è desunto (per compilazione) da quello del \texttt{.java}, solo espresso in un linguaggio diverso che garantisce la correttezza del contenuto e le performance di chi deve interpretarlo (JVM)
  }}
  \bl{\`E possibile ispezionare il contenuto di un \texttt{.class}}{\iz{
    \item Esempio di comando: \texttt{javap -v Counter.class}
    \item Comando in modalità \alert{verbose}
  }}
}

\fr{Classe \Cil{Counter}}{
  \sizedrangedcode{\scriptsize}{3}{30}{\ecl/classes/Counter.java}
}

\fr{Contenuto del classfile (1/2)}{
  \sizedcode{\tiny}{code/loading/out_1.txt}
}

\fr{Contenuto del classfile (2/2)}{
  \sizedcode{\tiny}{code/loading/out_2.txt}
}

\frs{5}{Caricamento delle classi e la JVM (1/2)}{
  \bl{Caricamento: chi?}{\iz{
    \item La JVM è un programma solitamente scritto in C/C++
    \item HotSpot di OpenJDK ($>$ 250K linee): \texttt{https://hg.openjdk.java.net/jdk/}
    \item Dispone di uno o più class-loader (realizzabili anche in Java dall'utente, estendendo \cil{java.lang.ClassLoader})
    \item Hanno il compito di cercare i classfile che servono, e di ``caricarli'' nella JVM
  }}
  \bl{Caricamento: quando?}{\iz{
    \item Tale caricamento **non** avviene necessariamente all'avvio: ogni classe viene caricata al momento del suo primo utilizzo!! (Schema \alert{by-need})
    \item Alla prima \cil{new}, o chiamata statica, o se serve una sottoclasse!
    \item (o con una richiesta esplicita come vedremo)
  }}
}

\fr{Caricamento delle classi e la JVM (2/2)}{
  \bl{Caricamento: da dove?}{\iz{
    \item Dal file system, attraverso il classpath e navigando i package
    \item Eventualmente dentro ai file JAR
    \item In alcune modalità, può caricare le classi anche via rete!
  }}
  \bl{Caricamento: cosa succede?}{\iz{
    \item La JVM prepara una opportuna struttura dati in memoria
    \item Inizializza i campi statici (e chiamando l'inizializzatore statico se definito)
  }}
  \bl{Nota: esiste anche l'inizializzatore ``non statico''}{\iz{
    \item viene richiamato all'atto della creazione dell'oggetto
  }}
}

\fr{Una parentesi: gli \Cil{initializer}}{
  \bl{Static initializer}{\iz{
    \item sintassi \cil{static\{...\}}
    \item eseguito ``subito'', prima di qualunque accesso alla classe
    \item in genere inizializza campi statici
    \item potrebbe eseguire altro codice di inizializzazione
  }}
  
  \bl{Non-static initializer}{\iz{
    \item sintassi \cil{\{...\}}
    \item eseguito prima di chiamare il costruttore
    \item in genere inizializza campi istanza
    \item potrebbe eseguire altro codice di inizializzazione
  }}
    
}

\fr{Inizializzatori all'opera}{
    \sizedrangedcode{\ssmall}{6}{100}{\ecl/classes/Initializers.java}
}

\fr{Ispezioniamo la dinamica di caricamento (e istanziazione)}{
    \sizedrangedcode{\ssmall}{3}{30}{\ecl/classes/Loading.java}
}

\fr{Perché questa gestione ``by-need'' del caricamento?}{
    \bl{Alcune motivazioni}{\iz{
	\item Permette alle applicazioni di ``partire'' più velocemente, in quanto non si carica tutto, solo quello che serve mano a mano
	\item In scenari di rete, consente di dover caricare da remoto solo sottoparti di applicazioni
	\item In scenari avanzati si potrebbero anche ``scaricare'' (togliere dalla JVM) le classi che sembrano non servire più, o addirittura fare ``hot-swapping'' (modifica di una classe)
	\item Alcune classi potrebbe essere aggiunte al volo per aumentare funzionalità senza spegnere l'applicazione
	\item \`E possibile ispezionare e usare il contenuto delle classi via \alert{reflection}
    }}
}

\section{Reflection API}

\fr{Reflection}{
    \bl{Packages \cil{java.lang} e \cil{java.lang.reflect}}{
    Forniscono una libreria che interagisce con la JVM per.. \iz{
	\item dare una rappresentazione ``ad oggetti'' del contenuto di una classe
	\item direttamente istanziare un oggetto, invocare metodi, accedere a campi
	\item forzare il caricamento di una classe
    }} 
    \bl{Sulla modalità ``via reflection''}{\`E più flessibile ma..\iz{
	\item è più lenta, anche di 1-2 ordini di grandezza (ma ottimizzabile)
	\item non interagisce con i controlli del compilatore (genera eccezioni..)
	\item pone problemi di security (che non analizzeremo in dettaglio..)
	\item consente di programmare accessi a classi non note a priori    
	\item[$\Rightarrow$] va usata di conseguenza.. quindi non abusarne
    }}
}

\fr{Motivazioni}{
    \bl{Tecniche messe a disposizione}{\iz{
	\item Unire classi non note ad una applicazione durante il suo funzionamento
	\item Trattare una stringa come identificatore del linguaggio
	\item Interagire con oggetti in modo dnamico, bypassando i test del compilatore
    }} 
    \bl{Applicazioni della reflection}{\iz{
	\item Estendibilità: Si può fare uso di classi esterne create/caricate ``al volo'', per modificare dinamicamente il comportamento di una applicazione
	\item Ambienti di sviluppo: Poter ispezionare la struttura di una classe o libreria
	\item Framework di Java: annotazioni, serializzazione, dynamic proxies,...
    }}
}

\frs{5}{La classe \Cil{java.lang.Class}}{
    \bl{Ogni suo oggetto rappresenta un tipo disponibile nella JVM}{
	\iz{
	    \item una classe (outer o inner, astratta o non), una interfaccia, una enumerazione, un array, i tipi primitivi e anche void
	    \item non i tipi generici (\cil{ArrayList<String>}) ma le classi generiche si
    }}
    \bl{Come si ottiene un oggetto di \cil{Class}? In tre modi:}{\iz{
	\item \cil{String.class}
	\item \cil{new String("this is a string").getClass()}
	\item \cil{Class.forName("java.lang.String")}
    }}   
    \bl{Genericità di \cil{Class}}{\iz{
      \item Solo via \cil{String.class} si può ottenere un oggetto di tipo \cil{Class<String>}, che avrà quindi metodi che ``sanno'' di lavorare su stringhe (ad esempio per creare oggetti di tipo stringa)
      \item Altrimeni restituisce un \cil{Class<?>}, e quindi prima o poi serviranno delle conversioni (con possibili cast ``unchecked'')
    }}
}

\fr{Esempi}{
    \sizedrangedcode{\tiny}{3}{100}{\ecl/classes/TryClass.java}
}

\fr{Esempi}{
    \sizedrangedcode{\ssmall}{3}{30}{\ecl/classes/UseClass.java}
}

\fr{Alcuni metodi di \Cil{java.lang.Class}}{
    \sizedcode{\tiny}{code/api/Class.java}
}

\fr{Alcuni metodi di \Cil{java.lang.reflect.Field}}{
    \sizedcode{\ssmall}{code/api/Field.java}
}

\fr{Alcuni metodi di \Cil{java.lang.reflect.Constructor}}{
    \sizedcode{\ssmall}{code/api/Constructor.java}
}

\fr{Alcuni metodi di \Cil{java.lang.reflect.Method}}{
    \sizedcode{\ssmall}{code/api/Method.java}
}

\fr{Esempio di chiamata dinamica di metodo}{
    \sizedrangedcode{\ssmall}{3}{30}{\ecl/classes/UseReflection.java}
}


\fr{Reflection e esecuzione ``dinamica''}{
   \bl{Applicazioni dinamiche}{\iz{
    \item Sono applicazioni che riescono ad eseguire codice aggiunto dinamicamente dopo che l'applicazione stessa è partita
    \item Questo tipo di meccanismo è molto importante in sistemi che non è possibile ``spegnere''
   }}
   \bl{Tecnica}{\iz{
    \item Un gestore dell'applicazione compila nuove classi e le aggiunge al CLASSPATH
    \item L'applicazione è già stata costruita in modo da accedere a certe classi da eseguire tramite il loro nome, eseguendone certi metodi
   }}
}

\fr{Caricamento ed esecuzione automatica}{
    %\sizedcode{\ssmall}{code/example/DynamicExecution.java}
    \sizedrangedcode{\tiny}{4}{36}{\ecl/classes/DynamicExecution.java}
}

\section{Un esempio di applicazione}

\fr{Sviluppiamo un semplice esempio}{
    \bl{Realizzazione automatizzata del \cil{toString} -- un mero esempio}{\iz{
	\item Scrivere i \cil{toString} è piuttosto noioso e ripetitivo
	\item Alcuni IDE (come Eclipse) lo generano automaticamente
	\item Come potremmo fornire un supporto programmato?
	\item Il principale problema è indicare dinamicamente, di volta in volta, come produrre la stringa sulla base delle proprietà di interesse
	\item Forniamo una soluzione base, facilmente estendibile dallo studente
    }}
    \bl{Idea}{\iz{
	\item Fornisco un metodo statico in una classe di funzionalità varie
	\item Accetta l'oggetto da stampare e una descrizione di cosa stampare
	\item Esempio: il nome della proprietà da ritrovare (assumendo ci sia un getter)
	\item Ritorna la stringa creata
    }}
}

\fr{\Cil{objectToString}}{
    \sizedrangedcode{\ssmall}{6}{50}{\ecl/classes/PrintObjectsUtilities.java}
}

\fr{Classi di prova}{
    \sizedcode{\scriptsize}{code/string/Person.java}
    \sizedcode{\scriptsize}{code/string/Teacher.java}
}


\begin{frame}[fragile]\frametitle{Uso di \Cil{objectToString}}
    \sizedrangedcode{\ssmall}{3}{50}{\ecl/classes/UsePrintObjectUtilities.java}
\begin{lstlisting}
Person
Person:  Name -> Mario
Person:  Name -> Mario | Id -> 101
Teacher:  Name -> Gino | Id -> 102 | Courses -> [PC, OOP]
\end{lstlisting}
\end{frame}

\section{Annotazioni}

\fr{Il Java annotation framework}{
    \bl{Annotazioni in Java}{\iz{
	\item Sono un meccanismo usato per ``annotare'' pezzi di codice
	\item Il compilatore di default ignora queste annotazioni
	\item A run-time, via reflection, è possibile verificare quali annotazioni e dove sono presenti
	\item \`E anche possibile istruire il compilatore a rigettare annotazioni mal formate
	\item Java fornisce alcune annotazioni standard
    }}
    \bl{Motivazioni}{\iz{
	\item Rendere il linguaggio più flessibile
	\item Consentire di realizzare piccole aggiunte ``programmate'' al linguaggio
    }}
}

\begin{frame}[fragile]\frametitle{Un primo esempio di annotazione: \texttt{\@Override}}
\begin{lstlisting}
class MyClass extends SuperClass{
    ...
    @Override
    public void myMethod(...){...}
\end{lstlisting}
\bl{Elementi principali}{\iz{
    \item Abbiamo annotato con \cil{\@Override} il metodo \cil{myMethod}
    \item \cil{javac} è istruito a rigettare questo codice se \cil{SuperClass} non definisce \cil{myMethod}{\iz{
	\item stessa cosa quando si implementa il metodo di una interfaccia
    }}
    \item Serve a evitare errori di nome nell'indicazione di \cil{myMethod}
    \item È prassi usarli sempre quando si fa overriding 
    \item Eclipse li aggiunge e segnala eventuali errori
}}
\end{frame}

\fr{Esempio \Cil{@Override}}{
    \sizedrangedcode{\footnotesize}{1}{50}{code/annotations/A.java}
}

\frs{5}{Altre annotazioni di libreria}{
    \bl{Cosa sono le annotazioni?}{\iz{
	\item Sono indicabili come sorta di interfacce
	\item Ogni package può esporre le proprie
	\item Le librerie di Java ne espongono varie
    }}
    \bl{\cil{java.lang.Override}}{\iz{
	\item Controllo statico del corretto overriding
    }}
    \bl{\cil{java.lang.SuppressWarnings}}{\iz{
	\item Dichiara del codice essere corretto: non genererà warning!
	\item Vuole come parametro il warning da disabilitare
    }}
    \bl{Altri}{\iz{
	\item \cil{java.lang.Deprecated}: marca ``deprecato'' il metodo
	\item \cil{java.lang.FunctionalInterface}: verifica che un solo metodo richieda implementazione
	\item Varie definite nel package \cil{java.lang.annotation}
    }}
}

\fr{Esempio \Cil{@SuppressWarnings}}{
    \sizedrangedcode{\ssmall}{1}{50}{code/annotations/Vector.java}
}

\fr{Annotazioni custom}{
    \bl{Definire le proprie annotazioni}{\iz{
	\item È possibile definire le proprie annotazioni, con una sintassi che ricalca molto da vicino quella delle interfacce
	\item È possibile via reflection capire quali metodi/campi/classi/costruttori sono stati annotati
    }}
    \bl{Dove si possono inserire?}{
	Per annotare la dichiarazione di classi, campi, metodi e costruttori\\
	Non è al momento consentito annotare anche i tipi mentre li si usano
    }
    \bl{Come li si dichiara}{\iz{
	\item L'uso di una annotazione genera un oggetto ispezionabile
	\item L'annotazione dichiara l'interfaccia di tale oggetto, e come lo si deve inizializzare
	\item Si forniscono anche informazioni ulteriori
    }}
}

\fr{Un esempio di applicazione}{
    \bl{Riprendiamo l'esempio \texttt{objectToString()}}{\iz{
	\item Costruiamo una gestione delle annotazioni che permetta di annotare alcuni metodi getter
	\item E costruiamo una nuova \cil{objectToString} che stampi controllando tali annotazioni
    }}
}

\fr{Dichiarazione \Cil{@ToString}}{
    \sizedrangedcode{\scriptsize}{1}{50}{\ecl/annotations/ToString.java}
}

\fr{Esempio di come si usa l'annotazione}{
    \sizedrangedcode{\scriptsize}{3}{50}{\ecl/annotations/Person.java}
}

\fr{Altro esempio}{
    \sizedrangedcode{\ssmall}{3}{50}{\ecl/annotations/Product.java}
}


\fr{Metodo \Cil{objectToString()}}{
    \sizedrangedcode{\tiny}{3}{50}{\ecl/annotations/PrintObjectsUtilities.java}
}

\fr{Uso \Cil{objectToString()}}{
    \sizedrangedcode{\scriptsize}{3}{50}{\ecl/annotations/UsePrintObjectsUtilities.java}
}


\section{JUnit}

\begin{frame}[fragile]\frametitle{Il framework per i collaudi software \Cil{JUnit}}
    \bl{Una applicazione delle annotazioni di Java... Idea:}{\iz{
	\item Creare delle classi di test, con metodi che realizzano degli scenari d'uso di certe classi da testare
	\item Tali metodi sono propriamente annotati
	\item Alla fine del loro lavoro tali metodi asseriscono se il risultato atteso è giusto
	\item Da Eclipse semplice esecuzione della classe come ``JUnit test'', dopo aver collegato la libreria Junit al ``build path'' del progetto
    }}
\end{frame}

\fr{Classe per il collaudo di un contatore}{
    \sizedrangedcode{\ssmall}{1}{50}{code/junit/CounterTest.java}
}

\begin{frame}[fragile]\frametitle{Altre modalità di asserire il risultato di un test}
Si veda: \texttt{https://github.com/junit-team/junit/wiki/Assertions}
\begin{lstlisting}[basicstyle=\ssmall\ttfamily]
// many things to import

public class AssertTests {
  @Test
  public void testAssertFalse() {
    org.junit.Assert.assertFalse("failure - should be false", false);
  }

  @Test
  public void testAssertSame() {
    Integer aNumber = Integer.valueOf(768);
    org.junit.Assert.assertSame("should be same", aNumber, aNumber);
  }

  // JUnit Matchers assertThat
  @Test
  public void testAssertThatEveryItemContainsString() {
    org.junit.Assert.assertThat(
        Arrays.asList(new String[] { "fun", "ban", "net" }), 
        everyItem(containsString("n")));
  }
  ..
}
\end{lstlisting}
\end{frame}

\fr{Testiamo il \Cil{RangeIterator}: e modifichiamo se serve..}{
    \sizedrangedcode{\ssmall}{3}{50}{\ecl/annotations/RangeIterator.java}
}

\fr{Possibile codice \Cil{JUnit}: 1/2}{
    \sizedrangedcode{\ssmall}{1}{28}{\ecl/annotations/TestRangeIterator.java}
}

\fr{Possibile codice \Cil{JUnit}: 2/2}{
    \sizedrangedcode{\tiny}{29}{100}{\ecl/annotations/TestRangeIterator.java}
}

\fr{Altro esempio: esercizio \Cil{Graph}}{
    \sizedrangedcode{\ssmall}{3}{50}{\ecl/exercise/Graph.java}
}

\fr{Un esempio di semplicissimo test}{
    \sizedrangedcode{\ssmall}{10}{50}{\ecl/exercise/TestGraphImpl.java}
}

\fr{Variante migliorativa (1/2)}{
    \sizedrangedcode{\ssmall}{11}{38}{\ecl/exercise/TestGraphImplImproved.java}
}

\fr{Variante migliorativa (2/2)}{
    \sizedrangedcode{\ssmall}{39}{50}{\ecl/exercise/TestGraphImplImproved.java}
}

\fr{Sul testing con JUnit}{
    \bl{In generale}{\iz{
        \item il testing è un aspetto fondamentale della software engineering
        \item il testing è un aspetto fondamentale della programmazione
        \item JUnit è uno strumento estremamente potente
        \item tutte queste questioni verranno affrontate alla magistrale
    }}
    \bl{Linee guida base per i vostri test}{\iz{
        \item ogni classe abbia una sua classe di test
        \item la classe di test sia scritta molto bene e sia comprensibile
        \item si abbiano metodi di test corti, specifici e con un buon nome
        \item si cerchi massima copertura dei test
    }}
}

\end{document}


