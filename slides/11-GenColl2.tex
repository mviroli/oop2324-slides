\documentclass[presentation]{beamer}
\usepackage{oop-slides}


\title[OOP11: Generici e Collezioni advanced]{11 \\ Generici e collezioni, pt 2}

\begin{document}

\newcommand{\ecl}{\eclipsepath{p11gencoll2}}

\frame[label=coverpage]{\titlepage}

\fr{Outline}{
  \bl{Goal della lezione}{\iz{
  \item Approfondire alcuni concetti sui generici
  \item Presentare altre classi per le collezioni
  }}
  \bl{Argomenti}{\iz{
  \item Il problema della type-erasure
  \item Polimorfismo vincolato
  \item Approfondimento sulle Wildcards
  \item Implementazioni di \cil{List} e \cil{Map}
  }}
}

\section{Il problema della type-erasure}

\fr{La classe generica \Cil{List}}{
  \sizedcode{\scriptsize}{09/code/List.java}
}

\fr{Uso di una classe generica}{
  \sizedcode{\scriptsize}{09/code/generics/UseList.java}
}

\fr{Terminologia, e elementi essenziali}{
  \bl{Data una classe generica \cil{C<X,Y>}..}{\iz{
    \item \cil{X} e \cil{Y} sono dette le sue \alert{type-variable}
    \item \cil{X} e \cil{Y} possono essere usate come un qualunque tipo dentro la classe (con alcune limitazioni che vedremo)
  }}
  \bl{I clienti delle classi generiche}{\iz{
    \item Devono usare \alert{tipi generici}: versioni ``istanziate'' delle classi generiche\iz{
      \item \cil{C<String,Integer>}, \cil{C<C<Object,Object>,Object>}
      \item Non \cil{C} senza parametri, altrimenti vengono segnalati dei warning
    }
    \item Ogni type-variable va sostituita con un tipo effettivo, ossia con un \alert{parametro}, che può essere{\iz{
      \item una classe (non-generica), p.e. \cil{Object}, \cil{String},..
      \item una type-variable definita, p.e. \cil{X,Y} (usate dentro la classe \cil{C<X,Y>})
      \item un tipo generico completamente istanziato, p.e. \cil{C<Object,Object>}
      \item ..o  parzialmente istanziato, p.e. \cil{C<Object,X>} (in \cil{C<X,Y>})
      \item NON con un tipo primitivo
    }}
  }}
}

\fr{Limitazioni all'uso dei generici}{
  \bl{Una type-variable (\cil{X}) non è usabile in istruzioni del tipo:}{\iz{
    \item \cil{new X()}, \cil{new X[]\{..\}}, \cil{new X[10]}, \cil{instanceof X}
    \item Il compilatore segnala un errore
    \item errore anche l'\cil{instanceof} su un tipo generico: \cil{o instanceof C<String>}
  }}
  \bl{Type-variable, e tipi generici danno warning se usati in situazioni ``borderline''}{\iz{
    \item... \cil{(X)o}, \cil{(C<String>)o}
    \item il compilatore segnala un ``unchecked warning''
  }}
  \bl{Perché queste limitazioni? (Odersky \& Runne \& Wadler, 1998)}{\iz{
    \item Derivano dallo schema di supporto ad erasure
    \item Il compilatore prende la classe \cil{List<X>} e la trasforma in qualcosa di simile alla classe \cil{ObjectList} prima di compilarla effettivamente..
    
  }}
}

\fr{Qualche prova}{
  \sizedrangedcode{\scriptsize}{3}{100}{\ecl/erasure/ErasurePitfalls.java}
}


\fr{La classe generica \Cil{Vector}}{
  \bl{Un dettaglio della sua implementazione:}{
  Per via della type erasure, il suo campo non può essere di tipo \cil{X[]}, bensì \cil{Object[]}
  }
  \sizedcode{\footnotesize}{09/code/Vector.java}
}

\fr{Uso di \Cil{Vector<X>}}{
  \sizedrangedcode{\scriptsize}{3}{100}{\eclipsepath{p09generics}/generics/UseVector.java}
}

\fr{Implementazione di \Cil{Vector} pt 1}{
\sizedrangedcode{\ssmall}{3}{26}{\eclipsepath{p09generics}/generics/Vector.java}
}

\fr{Implementazione di \Cil{Vector} pt 2}{
\sizedrangedcode{\ssmall}{27}{100}{\eclipsepath{p09generics}/generics/Vector.java}
}

\fr{Ancora sugli ``unchecked warning'' coi generici}{
\sizedrangedcode{\scriptsize}{3}{100}{\ecl/abstractions/ShowCast.java}
}

\section{Polimorfismo vincolato}

\fr{Polimorfismo vincolato}{
  \bl{Negli esempi visti finora..}{\iz{
    \item Data una classe \cil{C<X>}, \cil{X} può essere istanziato a qualunque sottotipo di \cil{Object}
    \item In effetti la definizione \cil{class C<X>\{..\}} equivale a \cil{class C<X extends Object>\{..\}}
  }}
  \bl{Polimorfismo vincolato (Igarashi \& Pierce \& Wadler, 2000)}{\iz{
    \item In generale invece, può essere opportuno vincolare in qualche modo le possibili istanziazioni di \cil{X}, ad essere sottotipo di un tipo più specifico di \cil{Object}
    \item \cil{class C<X extends D>\{..\}}
    \item In tal caso, dentro \cil{C}, si potrà assumere che gli oggetti di tipo \cil{X} rispondano ai metodi della classe \cil{D}
  }}
}

\fr{\Cil{LampsRow} generica: Definizione}{
  \sizedrangedcode{\ssmall}{3}{100}{\ecl/constrained/LampsRow.java}
}

\fr{Uso}{
  \bl{Motivazione per questa genericità. Ha senso se:}{\iz{
    \item si ritiene molto frequente l'uso di \cil{SimpleLamp} simili tra loro, ossia di una comune specializzazione (classe)
    \item è frequente l'uso di \cil{getLamp()}  e quindi del cast
  }}
  \sizedrangedcode{\scriptsize}{3}{100}{\ecl/constrained/UseLampsRow.java}
}

\section{Java Wildcards e sostituibilità}


\fr{Java Wildcards}{
  \sizedcode{\footnotesize}{09/code/Numbers.java}
  \sizedcode{\footnotesize}{09/code/Wildcards.java}
}

\fr{Java Wildcards}{
  \bl{3 tipi di wildcard (Igarashi \& Viroli, 2002)}{\iz{
    \item Bounded (covariante): \cil{C<? extends T>}{\iz{
      \item accetta un qualunque \cil{C<S>} con \cil{S <: T}
    }}
    \item Bounded (controvariante): \cil{C<? super T>}{\iz{
      \item accetta un qualunque \cil{C<S>} con \cil{S >: T}
    }}
    \item Unbounded: \cil{C<?>}{\iz{
      \item accetta un qualunque \cil{C<S>}
    }}    
  }}
  \bl{Uso delle librerie che dichiarano tipi wildcard}{\iz{
    \item Piuttosto semplice, basta passare un argomento compatibile o si ha un errore a tempo di compilazione
  }}
  \bl{Progettazione di librerie che usano tipi wildcard}{\iz{
    \item Molto avanzato: le wildcard pongono limiti alle operazioni che uno può eseguire, derivanti dal principio di sostituibilità
  }}
}

%\fr{Esempio Wildcard Bounded covariante: \Cil{C<? extends T>}}{
%  \sizedcode{\scriptsize}{09/code/Cov.java}
%}
%
%\fr{Esempio Wildcard Bounded controvariante: \Cil{C<? super T>}}{
%  \sizedcode{\scriptsize}{09/code/Contra.java}
%}
%
%\fr{Esempio Wildcard Unbounded: \Cil{C<?>}}{
%  \sizedcode{\scriptsize}{09/code/Unbounded.java}
%}


\fr{Approfondimento: sulla sostituibilità dei generici}{
  \bl{Domanda: \cil{Vector<Integer>} è un sottotipo di \cil{Vector<Object>}?}{
    Ovvero, possiamo pensare di passare un \cil{Vector<Integer>} in tutti i contesti in cui invece ci si aspetta un \cil{Vector<Object>}?
  }
  \bl{Risposta: no!! Sembrerebbe di si.. ma:}{
    cosa succede se nel metodo qui sotto passiamo un \cil{Vector<Integer>}?\\
    $\Rightarrow$ potremmo facilmente compromettere l'integrità del vettore
  }
  \sizedcode{\footnotesize}{09/code/Covariance.java}
}

\fr{Subtyping e safety}{
  \bl{\alert{Safety} di un linguaggio OO}{
    Se nessuna combinazione di istruzioni porta a poter invocare un metodo su un oggetto la cui classe non lo definisce{\iz{
      \item \`E necessario che il subtyping segua il principio di sostituibilità
    }}
    Più in generale, se non possono accadere errori a tempo di esecuzione..
  }
  \bl{Java}{\iz{
    \item Si pone dove possibile l'obbiettivo della safety
    \item Quindi, non è vero che \cil{Vector<Integer> <: Vector<Object>}
  }}
  \bl{Generici e safety}{
    In generale, istanziazioni diverse di una classe generica sono scollegate
    \iz{
    \item non c'è \alert{covarianza}: non è vero che \cil{C<T> <: C<S>} con \cil{T <: S}
    \item non c'è \alert{controvarianza}: non è vero che \cil{C<S> <: C<T>} con \cil{T <: S}
  }}
}

\fr{Unsafety con gli array di Java}{
  \bl{Gli array di Java sono trattati come covarianti!}{\iz{
    \item Gli array assomigliano moltissimo ad un tipo generico
    \item \cil{Integer[]} $\sim$ \cil{Array<Integer>}, \cil{T[]} $\sim$ \cil{Array<T>}
    \item E quindi sappiamo che non sarebbe safe gestirli con covarianza
    \item E invece in Java è esattamente così!! P.e. \cil{Integer[] <: Object[]}
    \item Quindi ogni scrittura su array potrebbe potenzialmente fallire\dots lanciando un \cil{ArrayStoreException}
  }}
  \sizedcode{\footnotesize}{09/code/Arrays.java}
}

\fr{Covarianza e operazioni di accesso}{
  \bl{La covarianza (\cil{C<T> <: C<S>} con \cil{T <: S}) sarebbe ammissibile se:}{\iz{
    \item La classe \cil{C<X>} non avesse operazioni che ricevono oggetti \cil{X}
    \item Ossia, ha solo campi privati e nessun metodo con argomento \cil{X}
  }}
  \bl{La controvarianza (\cil{C<S> <: C<T>} con \cil{T <: S}) sarebbe ammissibile se:}{\iz{
    \item La classe \cil{C<X>} non avesse operazioni che producono oggetti \cil{X}
    \item Ossia, ha solo campi privati e nessun metodo con tipo di ritorno \cil{X}
  }}
  \bl{In pratica:}{\iz{
    \item La maggior parte delle classi generiche \cil{C<X>} hanno campi di tipo \cil{X} (composizione) e operazioni getter e setter, e quindi per loro covarianza e controvarianza non funzionano
    \item Con le wildcard si aumenta il riuso dei metodi che non usano tutte le funzione del tipo generico in input
  }}
}


\fr{Esempio Wildcard Bounded covariante: \Cil{C<? extends T>}}{
  \sizedcode{\scriptsize}{09/code/variance/Cov.java}
}

\fr{Esempio Wildcard Bounded controvariante: \Cil{C<? super T>}}{
  \sizedcode{\scriptsize}{09/code/variance/Contra.java}
}

\fr{Esempio Wildcard Unbounded: \Cil{C<?>}}{
  \sizedcode{\scriptsize}{09/code/variance/Unbounded.java}
}

\fr{Un uso tipico: metodi in classi generiche}{
  \sizedcode{\scriptsize}{09/code/VectFrag.java}
  \sizedcode{\scriptsize}{09/code/variance/UseVector.java}
}


\fr{Un uso tipico: metodi in classi generiche}{
  \sizedcode{\scriptsize}{09/code/VectFrag.java}
  \sizedcode{\scriptsize}{09/code/variance/UseVector.java}
}


\section{Implementazioni di \cil{List}}

\fr{Implementazione collezioni -- UML}{
    \fg{height=0.75\textheight}{10/img/uml-abs.pdf}
}


\fr{\Cil{List}}{
  \sizedcode{\scriptsize}{10/code/short/List2.java}
}

\fr{Implementazioni di \Cil{List}}{
  \bl{Caratteristiche delle liste}{\iz{
    \item Sono sequenze: ogni elemento ha una posizione
    \item[$\rightarrow$] Il problema fondamentale è realizzare i metodi posizionali in modo efficiente, considerando il fatto che la lista può modificarsi nel tempo (altrimenti andrebbe bene un array)
  }}
  \bl{Approccio 1: \cil{ArrayList}}{
   Internamente usa un array di elementi con capacità maggiore di quella al momento necessaria. Se serve ulteriore spazio si alloca trasparentemente un nuovo e più grande array
  }
  \bl{Approccio 2: \cil{LinkedList}}{
   Usa una double-linked list. L'oggetto \cil{LinkedList} mantiene il riferimento al primo e ultimo elemento della lista, e alla dimensione della lista
  }
}

\fr{\Cil{ArrayList}}{
  \bl{Caratteristiche di performance}{\iz{
    \item Lettura/scrittura in data posizione sono a tempo costante
    \item La \cil{add()} è tempo costante ammortizzato, ossia, $n$ add si effettuano in $O(n)$
    \item Tutte le altre operazioni sono a tempo lineare
  }}
  \bl{Funzionalità aggiuntive}{Per migliorare le performance (e l'occupazione in memoria) in taluni casi l'utente esperto può usare funzioni aggiuntive\iz{
    \item Specificare la dim iniziale dell'array interno nella \cil{new}
    \item \cil{trimToSize()} e \cil{ensureCapacity()} per modifiche in itinere
  }}
}


\fr{\Cil{ArrayList}: aspetti aggiuntivi}{
  \sizedcode{\scriptsize}{10/code/short/ArrayList.java}
}

\fr{\Cil{UseArrayList}}{
  \sizedrangedcode{\ssmall}{5}{100}{\ecl/list/UseArrayList.java}
}

\fr{\Cil{LinkedList}}{
  \bl{Caratteristiche di performance}{\iz{
    \item Accesso e modifica in una data posizione hanno costo lineare
    \item Operazioni in testa o coda, quindi, sono a tempo costante
    \item Usa in generale meglio la memoria rispetto \cil{ArrayList}
    \item (di norma però si preferisce \cil{ArrayList})
  }}
  \bl{Caratteristiche aggiuntive}{\iz{
    \item Implementa anche l'interfaccia \cil{Queue}, che modella una coda (ad esempio FIFO) 
    \item Implementa anche l'interfaccia \cil{Deque} che estende \cil{Queue}, una coda a due estremi
  }}
}

\fr{\Cil{LinkedList}: funzioni aggiuntive relative a code (e stack)}{
  \sizedcode{\ssmall}{10/code/short/Queue.java}
  \sizedcode{\ssmall}{10/code/short/Deque.java}
}

\fr{\Cil{LinkedList}: costruzione}{
  \sizedcode{\ssmall}{10/code/short/LinkedList.java}
}

\fr{\Cil{UseLinkedList}}{
  \sizedrangedcode{\ssmall}{5}{100}{\ecl/list/UseLinkedList.java}
}



\section{Altre classi: \cil{Arrays} e \cil{Collections}}

\fr{Classi di utilità (moduli): \Cil{Arrays} e \Cil{Collections}}{
  \bl{\cil{java.util.Arrays}}{\iz{
    \item Contiene varie funzionalità d'ausilio alla gestione degli array
    \item In genere ha varie versione dei metodi per ogni array di tipo primitivo
    \item Ricerca binaria (dicotomica), Ordinamento (quicksort), copia
    \item Operazioni base (\cil{toString}, \cil{equals}, \cil{hashCode}), anche ricorsive
  }}
  \bl{\cil{java.util.Collections}}{\iz{
    \item Raccoglie metodi statici che sarebbero potuti appartenere alle varie classi/interfacce viste
    \item Ricerca binaria (dicotomica), Ordinamento (quicksort), copia, min, max, sublist, replace, reverse, rotate, shuffle
    \item Con esempi notevoli d'uso delle wilcard
  }}
}

\fr{\Cil{Arrays}: qualche esempio di metodi}{
  \sizedcode{\ssmall}{10/code/short/Arrays.java}
}

\fr{\Cil{UseArrays}: qualche esempio di applicazione}{
  \sizedrangedcode{\ssmall}{5}{100}{\ecl/functions/UseArrays.java}
}

\fr{\Cil{Collections}: qualche esempio di metodi}{
  \sizedcode{\ssmall}{10/code/short/Collections.java}
}

\fr{\Cil{UseCollections}: qualche esempio di applicazione}{
  \sizedrangedcode{\ssmall}{5}{100}{\ecl/functions/UseCollections.java}
}

\section{Il caso delle \cil{java.util.Map}}


\fr{JCF -- struttura semplificata}{
  \fg{height=0.75\textheight}{10/img/tax.png}
}

\fr{\Cil{Map}}{
  \sizedcode{\scriptsize}{11/code/short/Map_.java}
}

\fr{Usare le mappe}{
  \sizedrangedcode{\ssmall}{5}{100}{\ecl/map/UseMap.java}
}

\frs{5}{Due implementazioni di \Cil{Map} e \Cil{AbstractMap}}{
  \bl{\cil{Map<K,V>}}{\iz{
    \item Rappresenta una funzione dal dominio \cil{K} in \cil{V}
    \item La mappa tiene tutte le associazioni (o ``entry'')
    \item Non posso esistere due entry con stessa chiave (\cil{Object.equals})
    
  }}
  \bl{\cil{HashMap}}{\iz{
    \item Sostanzialmente un \cil{HashSet} di coppie \cil{Key}, \cil{Value}
    \item L'accesso ad un valore tramite la chiave è fatto con hashing
    \item Accesso a tempo costante, a discapito di overhead in memoria
  }}
  \bl{\cil{TreeMap}}{\iz{
    \item Sostanzialmente un \cil{TreeSet} di coppie \cil{Key}, \cil{Value}
    \item L'accesso ad un valore tramite la chiave è fatto con red-black tree
    \item Accesso in tempo logaritmico
    \item Le chiavi devono essere ordinate, come per i \cil{TreeSet}
  }}
}

\section{Un esercizio sulle collezioni}

\fr{Interfaccia da implementare}{
  \sizedrangedcode{\scriptsize}{3}{30}{\ecl/exercise/Graph.java}
}

\fr{Codice di prova}{
  \sizedrangedcode{\ssmall}{3}{30}{\ecl/exercise/UseGraph.java}
}

\fr{Strategia risolutiva}{ \bl{Passi:}{\en{
    \item Capire bene cosa la classe deve realizzare
    \item Pensare a quale tipo di collezioni può risolvere il problema in modo semplice e prestante
    \item Realizzare i vari metodi
    \item Controllare i casi particolari
}}
}


\end{document}

