\documentclass[presentation]{beamer}
\usepackage{../oop-slides}

\title[OOP15: GUI]{15 \\ Graphical User Interfaces}

\begin{document}

\newcommand{\ecl}{\idepath{p15gui}}

\frame[label=coverpage]{\titlepage}

\fr{Outline}{
  \bl{Goal della lezione}{\iz{
  \item Illustrare la libreria Java Swing
  \item Fornire pattern di progettazione per le GUI
  }}
  \bl{Argomenti}{\iz{
    \item Organizzazione della libreria Swing
    \item Panoramica dei meccanismi principali
    \item La gestione degli eventi nelle GUI
    \item Elementi di programmazione ad eventi
    \item Organizzazione MVC delle GUI
    \item \dots in seguito, JavaFX
  }}
}

\section{Introduzione}

\fr{Graphical User Interfaces (GUI)}{
  \bl{GUI}{\iz{
    \item Interfacce grafiche per l'interazione con l'utente
    \item Ritenute più semplici rispetto alle CUI (Console User Interfaces)
    \item Sfruttano la possibilità di disegnare più o meno arbitrariamente i pixel della matrice dello schermo
    \item Oltre allo schermo possono sfruttare altri dispositivi: mouse, tastiera,..
    \item Si appoggiano su astrazioni grafiche (pulsanti, icone, finestre)
  }}
  \bl{Gestione delle GUI in Java}{\iz{
    \item Abstract Window Toolkit (AWT) in Java 1 e 2 -- basso livello
    \item Java Swing in Java 5,6,7,8 
    \item Alternative: JavaFX (consigliata da Java 8), SWT (usato da Eclipse) 
    \item[$\Rightarrow$] Vedremo Swing, che si appoggia su AWT
    \item[$\Rightarrow$] In laboratorio vedrete la più moderna JavaFX
  }}
}

\fr{AWT e Swing: UML}{
    \fg{height=0.85\textheight}{img/swing.png}
}

\fr{AWT, Swing e concetti principali}{
    \bl{I due package}{\iz{
	\item \cil{java.awt}: Classi base e implementazioni supportate dal S.O.{\iz{
	    \item non molto utili da guardare in dettaglio
	    \item fornisce comunque l'architettura base
	}}
	\item \cil{javax.swing}: implementazioni gestite ``pixel per pixel''{\iz{
	    \item le classi \cil{J*} e quelle sottostanti
	}}
    }}
    \bl{Alcune classi base di Swing}{\iz{
        \item \cil{JFrame}: finestra con ``cornice'' (menù, barra, icone chiusura)
	\item \cil{JPanel}: pannello di componenti inseribili in un \cil{JFrame}
	\item \cil{JComponent}: componente (pulsante, textfield, ..)
	\item \cil{JDialog}: finestra di dialogo
	\item \cil{JWindow}: componente piazzabile nel desktop (senza cornice)
    }}
}

\fr{Concetti principali}{
    \fg{height=0.75\textheight}{img/swing-comps.pdf}
}


\fr{Un primo esempio}{
    \sizedrangedcode{\ssmall}{3}{100}{\ecl/examples/TrySwing.java}
    \backpick{4cm}{-2cm}{-1.5cm}{img/TrySwing.png}
}

\fr{Vari \Cil{JComponent} disponibili..}{
    \sizedrangedcode{\ssmall}{3}{100}{\ecl/examples/Components.java}
    \backpick{7cm}{-3.2cm}{-1.3cm}{img/Components.png}
}

\fr{Classi di Swing}{
    \fg{height=0.85\textheight}{img/swing-details.png}
}

\fr{Materiale da consultare}{
    \bl{Collezione di riferimenti utili}{\iz{
	\item JavaDoc delle librerie
	\item Tutorial ufficiali:{\iz{
	    \item \texttt{http://docs.oracle.com/javase/tutorial/uiswing/}
	}}
  }}
    \bl{Lezione di oggi}{\iz{
	\item Mostriamo le tecniche principali
	\item Occasionalmente mostreremo il funzionamento di vari componenti
	\item Costruire GUI efficaci (e avanzate) richiede però conoscenze ulteriori ottenibili all'occorrenza dai riferimenti di cui sopra
    }}
}

\section{Il layout dei pannelli}

\fr{Il problema del Layout di un pannello}{
    \bl{Problema}{\iz{
	\item Intervenire sulla politica di dislocazione dei componenti
	\item Scegliere politiche indipendenti dalle dimensioni della finestra
	\item Organizzare tali selezioni con una buona organizzazione OO
    }}
    \bl{La classe \cil{LayoutManager} e il pattern ``Strategy''}{\iz{
	\item Al pannello si passa un oggetto di \cil{LayoutManager}
	\item È lui che incapsula la strategia di inserimento dei componenti
	\item Vari casi: \cil{FlowLayout} (default), \cil{BorderLayout}, \cil{GridBagLayout},.. (tipicamente da comporre tra loro)
	\item Vedere: {\scriptsize \texttt{http://docs.oracle.com/javase/tutorial/uiswing/layout/visual.html}}
	\item Il metodo \cil{add()} di \cil{JPanel} accetta un ulteriore argomento (\cil{Object} o \cil{int}) usato dal Layout Manager
    }}
}

\fr{Senza Layout -- deprecabile}{
   \sizedrangedcode{\ssmall}{3}{100}{\ecl/examples/UseNoLayout.java}
   \backpick{5cm}{-2.2cm}{-1.2cm}{img/NoLayout.png}
}


\fr{\Cil{BorderLayout}}{
   \sizedrangedcode{\ssmall}{3}{100}{\ecl/examples/UseBorderLayout.java}
   \backpick{5cm}{-2.2cm}{-1.2cm}{img/BorderLayout.png}
}

\fr{Lavorare specializzando \Cil{JFrame}: \Cil{MyFrame}}{
   \sizedrangedcode{\ssmall}{3}{100}{\ecl/examples/MyFrame.java}
}

\fr{Nuova versione \Cil{UseBorderLayout2}}{
   \sizedrangedcode{\ssmall}{3}{100}{\ecl/examples/UseBorderLayout2.java}
   \backpick{5cm}{-2.2cm}{-1.2cm}{img/BorderLayout.png}
}

\fr{Qualche modifica: \Cil{UseBorderLayout3}}{
   \sizedrangedcode{\ssmall}{3}{100}{\ecl/examples/UseBorderLayout3.java}
   \backpick{5cm}{-2.2cm}{-1.2cm}{img/BorderLayoutBis.png}
}

\fr{\Cil{FlowLayout}}{
   \sizedrangedcode{\ssmall}{3}{100}{\ecl/examples/UseFlowLayout.java}
   \backpick{5cm}{-2.2cm}{-1.4cm}{img/FlowLayout.png}
}

\fr{Un uso combinato di \Cil{FlowLayout} e \Cil{BorderLayout}}{
   \sizedrangedcode{\ssmall}{3}{100}{\ecl/examples/UseFlowBorder.java}
   \backpick{4cm}{-1.8cm}{-2cm}{img/FlowBorder.png}

}

\fr{Altro uso combinato + \Cil{GridBagLayout}}{
   \sizedrangedcode{\tiny}{7}{100}{\ecl/examples/UseMixedLayouts.java}
   \backpick{4cm}{-1.8cm}{-1.2cm}{img/Mixed.png}
}

\fr{Lavorare incapsulando il frame dietro una interfaccia}{
   \bl{L'indipendenza dalla tecnologia delle GUI}{\iz{
    \item in applicazioni reali sarà importante costruire GUI in modo che le scelte di basso livello dettate da Swing siano ben nascoste
    \item così da tenere tutto il resto del sistema indipendente da Swing, e quindi ben organizzato e ad alto livello
   }}
   \bl{Tipica tecnica}{\iz{
    \item si disegni una \cil{interface} pulita per la GUI
    \item i dettagli implementativi al solito siano delegati ad una classe che implementa
   }}
}

\fr{Lavorare incapsulando il frame dietro una interfaccia}{
   \sizedrangedcode{\ssmall}{3}{100}{\ecl/gui/UserInterface.java}
   \sizedrangedcode{\tiny}{8}{100}{\ecl/gui/UserInterfaceImpl.java}
}

\fr{Lavorare incapsulando il frame dentro una classe}{
   \sizedrangedcode{\ssmall}{3}{100}{\ecl/gui/UseEncapsulatedFrame.java}
}


\section{La gestione degli eventi nelle GUI}

\fr{La gestione degli eventi}{
    \bl{Come rendere le interfacce ``vive''?}{\iz{
	\item come programmare la possibilità di intercettare le varie azioni che un utente potrebbe compiere sull'interfaccia, ossia la parte ``input''?
	\item al solito, sarebbe necessario uno strumento altamente configurabile e ben organizzato
    }}
    \bl{Il pattern Observer}{\iz{
	\item È possibile ``registrare'' nei componenti degli oggetti ``ascoltatori'' (\alert{listeners})
	\item Quando certi eventi accadono, il componente richiama un metodo dei listener registrati
	\item Tale metodo contiene il codice da eseguire in risposta all'evento
	\item Si assume (per ora) che tale codice arrivi ``velocemente'' a compimento
    }}
}

\fr{Il pattern \alert{Observer}}{
\fg{height=0.5\textheight}{img/observer.jpg}
\bx{\iz{
    \item \cil{Subject} è la sorgente degli eventi
    \item \cil{Observer} si registra con la \cil{attach(o:Observer)}
    \item Quando accade l'evento, \cil{Subject} chiama \cil{notify(e:Event)}
}}
}

\begin{frame}[fragile]\frametitle{Il caso dei click sui pulsanti: 3 classi}
\begin{lstlisting}[basicstyle=\footnotesize\ttfamily]
// E' il subject degli eventi
class JButton .. {
    void setActionCommand(String s){..}
    void addActionListener(ActionListener listener){..}
    void removeActionListener(ActionListener listener){..}
}

// Interfaccia da implementare per ascoltare gli eventi
interface ActionListener .. {
    void actionPerformed(ActionEvent e);
}

// Classe per rappresentare un evento
class ActionEvent .. {
    String	getActionCommand(){..}
    long	getWhen(){..}
    ..
}
\end{lstlisting}
\end{frame}

\fr{Catturare gli eventi dei pulsanti con \Cil{ActionCommand}}{
   \sizedrangedcode{\ssmall}{3}{100}{\ecl/examples/UseButtonEvents.java}
   \backpick{4cm}{-2.4cm}{-1.8cm}{img/actions.png}
}

\fr{Corrispondente listener, come classe esterna}{
   \sizedrangedcode{\scriptsize}{3}{100}{\ecl/examples/MyActionListener.java}
}

\fr{Versione incapsulata (inner listener + source eventi)}{
   \sizedrangedcode{\tiny}{9}{100}{\ecl/examples/EventsFrame.java}
}


\fr{Listeners come classi anonime}{
   \sizedrangedcode{\ssmall}{3}{100}{\ecl/examples/UseButtonEvents2.java}
}

\fr{Panoramica eventi-listeners}{
    \fg{height=0.75\textheight}{p15gui/examples/events.png}
}

\fr{Sulla gestione degli eventi}{
    \bl{Il flusso di controllo con le GUI di Swing}{\iz{
	\item Quando si crea un \cil{JFrame}, la JVM crea l'\cil{EventDispatchThread} (EDT)
	\item Quindi l'applicazione non termina quando il \cil{main} completa
	\item Quando un evento si verifica la JVM fa eseguire il corrispondente codice all'EDT
	\item Ecco perché la GUI non risponde a nuovi eventi finché uno precedente non è stato gestito
	\item Per gestire con migliore flessibilità le GUI servono meccanismi di programmazione concorrente, che vedremo in futuro
    }}    
}



\section{Alcune funzionalità avanzate GUI}

\fr{GUI con I/O: listeners che modificano l'interfaccia}{
   \sizedrangedcode{\ssmall}{3}{100}{\ecl/examples/UseIOGUI.java}
   \backpick{5cm}{-2.2cm}{-2cm}{img/iogui.png}
}

\fr{GUI con Layout dinamico}{
   \sizedrangedcode{\ssmall}{3}{100}{\ecl/examples/UseDynamicLayout.java}
   \backpick{5cm}{-2.2cm}{-2cm}{img/adding.png}
}

\fr{Uso di un pannello come Canvas}{
   \sizedrangedcode{\tiny}{8}{100}{\ecl/examples/UseCanvas.java}
   \backpick{4cm}{-1.8cm}{-1.2cm}{img/Canvas.png}
}

\fr{La classe \Cil{DrawPanel}}{
   \sizedrangedcode{\tiny}{7}{100}{\ecl/examples/DrawPanel.java}
}

\fr{Uso delle finestre di dialogo}{
   \sizedrangedcode{\tiny}{8}{100}{\ecl/examples/UseDialogs.java}
   \backpick{4cm}{-1.8cm}{-1.3cm}{img/dialogs.png}
}

\fr{Le GUI builder}{
    \bl{Cosa sono?}{\iz{
	\item Sono sistemi software usabili per creare il codice che genera le interfacce
	\item Permettono una descrizione WYSIWYG (What you see is what you get)
	\item Spesso non sono particolarmente semplici da  usare
	\item Con un po' di esperienza e una buona conoscenza delle librerie sottostanti, possono essere usati con successo
	\item Se li si usasse, si deve però anche comprendere (e criticare) il codice che producono
    }}
}

\fr{WindowBuilder: un plugin per Eclipse}{
    \fg{height=0.75\textheight}{img/builder.png}
}


\section{Organizzazione applicazioni grafiche con MVC}

\fr{Quale architetturale complessiva?}{
  \bl{Cos'è una architettura software?}{\iz{
    \item è la struttura del sistema software di interesse
    \item definisce: {\en{ 
        \item elementi software (o componenti, che poi diventeranno (gruppi di) classi/interfacce)
        \item relazioni fra elementi (tipicamente, dipendenze/usi)
        \item proprietà degli elementi software (ruoli/task) e delle relazioni (tipologia, dati/flussi coinvolti)
  }}}}
  \bl{Pattern architetturale}{\iz{
    \item è una architettura software ritenuta ricorrente e con benefici, e quindi riusabile
    \item è descritta generalmente, e quindi è da calare nel contesto specificio del software da realizzare
    \item esempi: \alert{MVC}, MVVM, ECB, Layers, P2P
  }}
}

\fr{Il pattern architetturale MVC}{
  \bl{MVC -- divide l'applicazione in 3 parti}{\iz{
    \item \cil{Model}: modello OO del dominio applicativo del sistema
    \item \cil{View}: gestisce le interazioni con l'utente (input e output)
    \item \cil{Controller}: gestisce il coordinamento fra Model e View
  }}
  \fg{height=0.6\textheight}{img/mvc.png}
}



\frs{5}{Applicazione del MVC}{
    \bl{Sulla costruzione di applicazioni con GUI}{\iz{
	\item Specialmente se non esperti, possono essere alquanto laboriose
	\item Usare un approccio strutturato sembra richiedere più tempo nel complesso, ma in realtà porta a soluzione più facilmente modificabili e controllabili
    }}
    \bl{Alcune linee guida}{\iz{
	\item Usare il pattern MVC per la struttura generale
	\item Identificate le varie ``interazioni'', e quindi costruire le interfacce dei 3 componenti bene fin dall'inizio
	\item Cercare massima indipendenza fra i vari componenti (interfacce con meno metodi possibile)
	\item La tecnologia di V non abbia dipendenze in C e M (e viceversa)
	\item Costruire e testare modello e GUI separatamente (M e V), poi collegare il tutto col controllore (C) che risulterà particolarmente esile
    }}
}

\fr{MVC con le GUI: un esempio di design}{
    \fg{height=0.55\textheight}{p15gui/mvc/MVC-abstract.png}
    model, view e controller potrebbero poi delegare a varie classi aggiuntive\dots 
}

\fr{Componenti e loro interazioni}{
    \bl{MVC}{\iz{
	\item \cil{Model}: incapsula dati e logica relativi al dominio della applicazione
	\item \cil{View}: incapsula la GUI, le sue sottoparti, e la logica di notifica
	\item \cil{Controller}: intercetta gli eventi della View, comanda le modifiche al modello, cambia di conseguenza la View
    }}
    \bl{Interfacce -- nomi da modificare in una applicazione concreta}{\iz{
	\item \cil{ModelInterface}: letture/modifiche da parte del Controller
	\item \cil{ViewObserver}: comandi inviati dalla view al controller (\cil{void})
	\item \cil{ViewInterface}: comandi per settare la view, notifiche a fronte dei comandi (errori..)
    }}
}

\fr{Un esempio di applicazione: \Cil{DrawNumber}}{
    \fg{height=0.5\textheight}{\ecl/mvc/class.png}
}

\fr{Interfaccia del model: \Cil{DrawNumber}}{
\sizedrangedcode{\scriptsize}{3}{100}{\ecl/mvc/model/DrawNumber.java}
\sizedrangedcode{\scriptsize}{3}{100}{\ecl/mvc/model/DrawResult.java}
\sizedrangedcode{\scriptsize}{3}{100}{\ecl/mvc/model/AttemptsLimitReachedException.java}
}

\fr{Implementazione del model: \Cil{DrawNumberImpl} (1/2)}{
\sizedrangedcode{\ssmall}{5}{25}{\ecl/mvc/model/DrawNumberImpl.java}
}

\fr{Implementazione del model: \Cil{DrawNumberImpl} (2/2)}{
\sizedrangedcode{\ssmall}{26}{100}{\ecl/mvc/model/DrawNumberImpl.java}
}

\fr{Interfacce della view: \Cil{DrawNumberView}}{
\sizedrangedcode{\scriptsize}{5}{100}{\ecl/mvc/view/DrawNumberView.java}
\sizedrangedcode{\scriptsize}{3}{100}{\ecl/mvc/view/DrawNumberViewObserver.java}
}

\fr{Implementazione della view: \Cil{DrawNumberViewImpl} (1/3)}{
\sizedrangedcode{\ssmall}{14}{40}{\ecl/mvc/view/DrawNumberViewImpl.java}
}

\fr{Implementazione della view: \Cil{DrawNumberViewImpl} (2/3)}{
\sizedrangedcode{\ssmall}{41}{68}{\ecl/mvc/view/DrawNumberViewImpl.java}
}

\fr{Implementazione della view: \Cil{DrawNumberViewImpl} (3/3)}{
\sizedrangedcode{\ssmall}{70}{110}{\ecl/mvc/view/DrawNumberViewImpl.java}
}



\fr{Implementazione del controller: \Cil{DrawNumberApp} (1/2)}{
\sizedrangedcode{\ssmall}{6}{33}{\ecl/mvc/controller/DrawNumberApp.java}
}

\fr{Implementazione del controller: \Cil{DrawNumberApp} (2/2)}{
\sizedrangedcode{\ssmall}{34}{100}{\ecl/mvc/controller/DrawNumberApp.java}
}

\fr{Linee guida generali consigliate su MVC}{
   \bl{Metodologia proposta}{\iz{
      \item progettare le 3 interfacce{\iz{
         \item M: metodi di ``dominio'', chiamati da C
         \item C: metodi (\cil{void}) chiamati da V, esprimono ``azioni utente''
         \item V: metodi (\cil{void}) chiamati da C, esprimono richieste di visualizzazione
      }}
      \item la tecnologia scelta per le GUI sia interna alla View, e mai menzionata altrove o nelle interfacce
      \item implementare separatamente M, V e C, poi comporre e testare
      \item tipicamente: M, V e C si compongono di varie sottoparti
   }}
   \bl{Aspetti}{\iz{ 
      \item MVC è implementato in vari modi 
      \item l'approccio proposto è particolarmente indicato per la sua semplicità
      \item si usino altri approcci se non peggiorativi
   }}
}

\end{document}


\fr{Un esempio di applicazione}{
    \fg{height=0.65\textheight}{img/person-editor.png}
}

\fr{Funzionalità della applicazione}{
    \bl{Funzionalità lato utente}{\iz{
	\item Editor di dati di persone
	\item Load/save su file del set di persone
	\item Comando add apre una dialog
    }}
    \bl{Alcuni dettagli interni}{\iz{
	\item Il modello estende \cil{HashSet<Person>}, e incapsula la logica di load/save
	\item La dialog invia i dati al controllore che ne controlla la correttezza attraverso il modello
	\item Gli errori nel modello causano una interazione fra controllore e view per mostrare dialog d'errore
	\item Il Frame disegnato via codice, la dialog con WindowBuilder (esperienza positiva!)
    }}
}

\fr{UML più dettagliato}{
    \vspace{-20pt}\fg{height=0.8\textheight}{p15gui/mvc/MVC.png}
}

\section{Elementi di programmazione ad eventi}

\fr{Programmazione sequenziale vs event-based}{
    \bl{Programmazione sequenziale}{\iz{
        \item Un programma esegue passo-passo delle istruzioni, e controlla di quando in quando (bloccandosi o meno) se ha ricevuto input dall'esterno, e di quando in quando produce degli output
        \item È lo stile tradizionale della computazione ``algoritmica''
    }}    
    \bl{Programmazione ad eventi}{\iz{
        \item All'arrivo di un evento, uno (o più) handler corrispondente viene eseguito fino alla sua terminazione, eventualmente (non necessariamente) producendo un output
        \item È lo stile della computazione ``interattiva'', usata tipicamente con le GUI, i sistemi Web, i sistemi embedded, i sistemi robotici
    }}
}

\fr{Implementiamo una mini-libreria per il pattern observer}{
    \bl{Classe \cil{ESource} e interfaccia \cil{EObserver}}{\iz{
        \item Un \cil{ESource} è un generatore di eventi
        \item Accetta registrazioni da uno o più \cil{EObserver}
        \item \cil{EObserver} ha un metodo \cil{update} per ricevere gli eventi
        \item \cil{ESource} fornisce un metodo per notificarli tutti
    }}
    \bl{Esempio di applicazione ad eventi}{\iz{
        \item Leggiamo da tastiera un insieme di numeri positivi
        \item All'arrivo di un numero negativo vogliamo stampare la somma dei numeri ottenuti fin lì sia su tastiera che su file (in generale su un qualunque stream)
        \item Concepiamo il sistema ad ``eventi''
    }}    
}

\fr{Diagramma delle classi}{
    \fg{height=0.55\textheight}{p15gui/events/events.png}
    \bl{Idea: 4 oggetti}{\iz{
        \item un \cil{KbdInputSource} che genera un evento per ogni intero digitato
        \item un \cil{Adder} che genera l'evento ``somma'' quando giunge un negativo
        \item due \cil{ToStream} che ricevono quest'ultimo e lo stampano
    }}
}

\fr{Classe di partenza}{
    \sizedrangedcode{\ssmall}{3}{100}{\ecl/events/Application.java}
}

\fr{\Cil{EObserver} e \Cil{ESource}}{
    \sizedrangedcode{\ssmall}{3}{100}{\ecl/events/EObserver.java}
    \sizedrangedcode{\ssmall}{3}{100}{\ecl/events/ESource.java}
}

\fr{\Cil{KbdInputSource}}{
    \sizedrangedcode{\ssmall}{3}{100}{\ecl/events/KbdInputSource.java}
}

\fr{\Cil{Adder}}{
    \sizedrangedcode{\ssmall}{3}{100}{\ecl/events/Adder.java}
}

\fr{\Cil{ToStream}}{
    \sizedrangedcode{\ssmall}{3}{100}{\ecl/events/ToStream.java}
}


\fr{Considerazioni sulla programmazione ad aventi}{
    \bl{Aspetti positivi}{\iz{
        \item Molto utile quando si gestisce l'interazione col SISOP
        \item Consente altissimo disaccoppiamento fra i componenti
        \item Consente alta flessibilità nella composizione dinamica
        \item Se ben realizzata porta a buone performance 
    }}
    \bl{Aspetti negativi}{\iz{
        \item Più sofisticata e meno intuitiva
        \item Più difficile intercettare errori di programmazione
        \item Più laboriosa
        \item Potrebbe scomodare aspetti di multi-threading
    }}    
}
