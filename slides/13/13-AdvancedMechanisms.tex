\documentclass[presentation]{beamer}
\usepackage{../oop-slides}


\title[OOP13: Meccanismi Avanzati]{13 \\ Meccanismi Avanzati \\ Classi innestate e enumerazioni}

\begin{document}

\newcommand{\ecl}{\idepath{p13advanced}}
\frame[label=coverpage]{\titlepage}

\fr{Outline}{
  \bl{Goal della lezione}{\iz{
  \item Illustrare meccanismi avanzati della programmazione OO
  \item Dare linee guida sul loro utilizzo
  }}
  \bl{Argomenti}{\iz{
  \item Enumerazioni
  \item Classi innestate statiche
  \item Inner class
  \item Classi locali
  \item Classi anonime
  \item Mappe del Collection Framework
  }}
}

\section{Classi innestate statiche}

\begin{frame}[fragile]\frametitle{Classi innestate statiche -- idea e terminologia}
  \bl{Principali elementi}{\iz{
    \item Dentro una classe \cil{A}, chiamata \alert{outer} è possibile innestare la definizione di un'altra classe \cil{B}, chiamata \alert{innestata (statica)} -- in inglese, \alert{static nested}
    \item \cil{B} viene quindi vista come se fosse un membro \cil{statico} di \cil{A} (richiamabile via \cil{A}, come: tipo, per le \cil{new} e le chiamate statiche)
  }}
\begin{lstlisting}
// situazione di partenza
class A {...}
class B {...}
\end{lstlisting}
\begin{lstlisting}
// modifica, usando le inner class
class A {
    ...
    static class B { .. }
}
\end{lstlisting}
\end{frame}

\begin{frame}[fragile]\frametitle{Classi innestate statiche -- casistica}
  \bl{Possibilità di innestamento}{\iz{
    \item Anche una interfaccia può fungere da \cil{Outer}
    \item Si possono innestare anche interfacce
    \item Il nesting può essere multiplo e/o multilivello
    \item L'accesso alle classi/interfacce innestate statiche avviene con sintassi \cil{Outer.A}, \cil{Outer.B}, \cil{Outer.I}, \cil{Outer.A.C}
  }}
\begin{lstlisting}
class Outer {
    ...
    static class A { .. static class C{..} ..}
    static class B {..}
    interface I {..} // static è implicito
}
\end{lstlisting}
\end{frame}

\begin{frame}[fragile,shrink=5]\frametitle{Classi innestate statiche -- accesso}
  \bl{Uso}{\iz{
    \item L'accesso alle classi/interfacce innestate statiche avviene con sintassi \cil{Outer.StaticNested}
    \item Da dentro \cil{Outer} si può accedere anche direttamente con \cil{StaticNested}
    \item L'accesso da fuori \cil{Outer} di \cil{StaticNested} segue le regole del suo modificatore d'accesso
    \item Esterna e interna si vedono mutuamente i membri \cil{private} {\iz{
        \item ossia il significato completo di \cil{private} è: 
        \item ``privato a livello della outerclass più esterna'' (vista come unità di design)
    }}
  }}
\begin{lstlisting}
class Outer {
    ...
    static class StaticNested { 
       ...
    }
}
..
Outer.StaticNested obj = new Outer.StaticNested(...);
\end{lstlisting}
\end{frame}

\begin{frame}\frametitle{Motivazioni}
  \bl{Una necessità generale}{
    Vi sono situazioni in cui per risolvere un singolo problema è opportuno generare più classi, e si reputa sconveniente usare sorgenti diversi
  }
  \bl{Almeno tre motivazioni (non necessariamente contemporanee)}{\iz{
    \item Evitare il proliferare di classi in un package, specialmente quando poche di queste debbano essere pubbliche
    \item Migliorare l'incapsulamento, con un meccanismo per consentire un accesso locale anche a membri \cil{private}
    \item Migliorare la leggibilità, inserendo classi là dove serve (con nomi qualificati, quindi più espressivi)
  }}
\end{frame}

\fr{Caso 1}{
    \bl{Specializzazioni come classi innestate}{\iz{
	\item La classe astratta, o comunque base, è la outer
	\item Alcune specializzazioni ritenute frequenti e ovvie vengono innestate, ma comunque rese pubbliche
	\item due implicazioni: {\iz{
	  \item schema di nome delle inner class
	  \item possibilità di accedere ai membri statici
	}}
    }}
    \bl{Esempio}{\iz{
	\item \cil{Counter}, \cil{Counter.Bidirectional}, \cil{Counter.Multi}
    }}
    \bl{Note}{
        Un sintomo della possibilità di usare le classi nested per questo caso è quando ci si trova a costruire classi diverse costuite da un nome composto con una parte comune (\cil{Counter}, \cil{BiCounter}, \cil{MultiCounter})
    }
}

\fr{Classe \Cil{Counter} e specializzazioni innestate (1/2)}{
  \sizedcode{\scriptsize}{code/nested/Counter_.java}
}

\fr{Classe \Cil{Counter} e specializzazioni innestate (2/2)}{
  \sizedcode{\scriptsize}{code/nested/Counter_2.java}
}

\fr{Uso di \Cil{Counter} e specializzazioni innestate}{
  \sizedrangedcode{\scriptsize}{5}{100}{\ecl/nested/UseCounter.java}
}

\fr{Caso 2}{
    \bl{Necessità di una classe separata ai fini di ereditaerità}{
    In una classe potrebbero servire sotto-comportamenti che debbano:
    \iz{
	\item implementare una data interfaccia
	\item estendere una data classe
    }}
    \bl{Esempio}{\iz{
	\item \cil{Range}, \cil{Range.Iterator}
    }}
    \bl{Nota}{
    In tal caso spesso tale classe separata non deve essere visibile dall'esterno, quindi viene indicata come \cil{private}
    }
}

\fr{Classe \Cil{Range} e suo iteratore (1/2)}{
  \sizedcode{\scriptsize}{code/nested/Range_.java}
}

\fr{Classe \Cil{Range} e suo iteratore (2/2)}{
  \sizedcode{\scriptsize}{code/nested/Range_2.java}
}

\fr{Uso di \Cil{Range}}{
  \srcode{\scriptsize}{3}{30}{\ecl/nested/UseRange.java}
}

\fr{Caso 3}{
    \bl{Necessità di comporre una o più classi diverse}{\iz{
	\item Ognuna realizzi un sotto-comportamento
	\item Per suddividere lo stato dell'oggetto
	\item Tali classi non utilizzabili indipendentemente dalla outer
    }}
    \bl{Esempio tratto dal Collection Framework}{\iz{
	\item \cil{Map}, \cil{Map.Entry}
	\item (una mappa è ``osservabile'' come set di entry)
    }}
}

\fr{Riassunto classi innestate statiche}{
    \bl{Principali aspetti}{\iz{
	\item Da fuori (se pubblica) vi si accede con nome \cil{Outer.StaticNested}
	\item \cil{Outer} e \cil{StaticNested} sono co-locate: si vedono i membri \cil{private}
    }}
    \bl{Motivazione generale}{\iz{
	\item Voglio evitare la proliferazione di classi nel package
	\item Voglio sfruttare l'incapsulamento
    }}
    \bl{Motivazione per il caso \cil{public}}{\iz{
	\item Voglio enfatizzare i nomi \cil{Out.C1}, \cil{Out.C2},..
    }}
    \bl{Motivazione per il caso \cil{private} -- è il caso più frequente}{\iz{
      \item Voglio realizzare una classe a solo uso della outer, invisibile alle altre classi del package
    }}
}




\section{Il caso delle \cil{java.util.Map}}


\fr{JCF -- struttura semplificata}{
  \fg{height=0.75\textheight}{../10/img/tax.png}
}

\fr{\Cil{Map}}{
  \sizedcode{\scriptsize}{code/short/Map_.java}
}

%\fr{Usare le mappe}{
%  \sizedcode{\ssmall}{code/map/UseMap.java}
%}

\fr{Implementazione mappe -- UML}{
    \fg{height=0.75\textheight}{img/uml-abs.pdf}
}

\fr{\Cil{Map.Entry}}{
\bl{Ruolo di \cil{Map.Entry}}{\iz{
    \item Una mappa può essere vista come una collezione di coppie chiave-valore, ognuna incapsulata in un \cil{Map.Entry}
    \item Quindi, una mappa è composta da un set di \cil{Map.Entry}
}}
  \sizedcode{\scriptsize}{code/short/Map_2.java}
}


\fr{Uso di \Cil{Map.Entry}}{
  \sizedrangedcode{\ssmall}{5}{100}{\ecl/map/UseMap.java}
}

\section{Inner Class}

\begin{frame}[fragile,shrink=5]\frametitle{Inner Class -- idea}
  \bl{Principali elementi}{\iz{
    \item Dentro una classe \cil{Outer}, è possibile innestare la definizione di un'altra classe \cil{InnerClass}, senza indicazione \cil{static}!
    \item \cil{InnerClass} è vista come se fosse un membro \alert{non-statico} di \cil{Outer} al pari di altri campi o metodi, quindi da richiamare (ad esempio quando si fa una \cil{new}) su una istanza di \cil{Outer}, chiamata ``enclosing instance''
    \item L'effetto è che una istanza di \cil{InnerClass} ha sempre un riferimento ad una enclosing instance accessibile con la sintassi \cil{Outer.this}, che ne rappresenta il \alert{contesto}
  }}
\begin{lstlisting}
class Outer {
    ...
    class InnerClass { // Nota.. non è static!
        ...
        // ogni oggetto di InnerClass avrà un riferimento ad
        // un oggetto di Outer, denominato Outer.this
    }
}
\end{lstlisting}
\end{frame}

\fr{Un semplice esempio}{
  \sizedrangedcode{\ssmall}{3}{100}{\ecl/nested/Outer.java}
}

\fr{Uso di \Cil{Inner} e \Cil{Outer}}{
  \sizedrangedcode{\scriptsize}{3}{100}{\ecl/nested/UseOuter.java}
}

\fr{Enclosing instance -- istanza esterna}{
    \bl{Gli oggetti delle inner class}{\iz{
	\item Sono creati con espressioni: \cil{<obj-outer>.new <classe-inner>(<args>)}
	\item (la parte \cil{<obj-outer>} è omettibile quando sarebbe \cil{this})
	\item Possono accedere all'enclosing instance con notazione \cil{<classe-outer>.this}
    }}
    \bl{Motivazioni: quelle relative alle classi innestate statiche, più..}{\iz{
	\item ...quando è necessario che ogni oggetto inner tenga un riferimento all'oggetto outer
	\item pragmaticamente: usato quasi esclusivamente il caso \cil{private}
    }}
    \bl{Esempio}{\iz{
	\item La classe \cil{Range} già vista usa una static nested class, che però ben usufruirebbe del riferimento all'oggetto di \cil{Range} che l'ha generata
    }}
}

\fr{Una variante di \Cil{Range}}{
  \sizedrangedcode{\tiny}{3}{100}{\ecl/nested/Range2.java}
}

\section{Classi locali}

\begin{frame}[fragile,shrink=5]\frametitle{Classi locali -- idea}
  \bl{Principali elementi}{\iz{
    \item Dentro un metodo di una classe \cil{Outer}, è possibile innestare la definizione di un'altra classe \cil{LocalClass}, senza indicazione \cil{static}!
    \item La \cil{LocalClass} è a tutti gli effetti una inner class (e quindi ha enclosing instance)
    \item In più, la \cil{LocalClass} ``vede'' anche le variabili nello scope del metodo in cui è definita, \alert{usabili solo se final}, o se ``di fatto finali''
  }}
\begin{lstlisting}
class Outer {
    ...
    void m(final int x){
        final String s=..;
        class LocalClass { // Nota.. non è static!
        ... // può usare Outer.this, s e x
        }
        LocalClass c=new LocalClass(...);
    }
}
\end{lstlisting}
\end{frame}

\fr{\Cil{Range} tramite classe locale}{
  \sizedrangedcode{\tiny}{3}{100}{\ecl/nested/Range3.java}
}

\fr{Classi locali -- motivazioni}{
    \bl{Perché usare una classe locale invece di una inner class}{\iz{
	\item Tale classe è necessaria solo dentro ad un metodo, e lì la si vuole confinare
	\item \`E eventualmente utile accedere anche alle variabili del metodo
    }}
    \bl{Pragmaticamente}{\iz{
      \item Mai viste usarle.. si usano invece le classi anonime..
    }}
}

\section{Classi anonime}

\begin{frame}[fragile,shrink=5]\frametitle{Classi anonime -- idea}
  \bl{Principali elementi}{\iz{
    \item Con una variante dell'istruzione \cil{new}, è possibile innestare la definizione di un'altra classe senza indicarne il nome
    \item In tale definizione non possono comparire costruttori
    \item Viene creata al volo una classe locale, e da lì se ne crea un oggetto
    \item Tale oggetto, come per le classi locali, ha enclosing instance e ``vede'' anche le variabili \alert{final} (o di fatto finali) nello scope del metodo in cui è definita
  }}
\begin{lstlisting}
class C {
    ...
    Object m(final int x){
        return new Object(){
             public String toString(){ return "Valgo "+x; }
        }
    }
}
\end{lstlisting}
\end{frame}

\fr{\Cil{Range} tramite classe anonima -- la soluzione ottimale}{
  \sizedrangedcode{\tiny}{3}{100}{\ecl/nested/Range4.java}
}

\fr{Classi anonime -- motivazioni}{
    \bl{Perchè usare una classe anonima?}{\iz{
	\item Se ne deve creare un solo oggetto, quindi è inutile anche solo nominarla
	\item Si vuole evitare la proliferazione di classi
	\item Tipicamente: per implementare ``al volo'' una interfaccia
    }}
}

\fr{Altro esempio: classe anonima da  \Cil{Comparable}}{
   \sizedrangedcode{\scriptsize}{5}{100}{\ecl/nested/UseSort.java}
}

\fr{Riassunto e linee guida}{
    \bl{Inner class (e varianti)}{
	Utili quando si vuole isolare un sotto-comportamento in una classe a sé, senza dichiararne una nuova che si affianchi alla lista di quelle fornite dal package, ma stia ``dentro'' una classe più importante
    }
    \bl{Se deve essere visibile alle altre classi}{\iz{
	\item Quasi sicuramente, una static nested class
    }}
    \bl{Se deve essere invisibile da fuori}{\iz{
	\item Si sceglie uno dei quattro casi a seconda della visibilità che la inner class deve avere/dare{\iz{
	    \item static nested class: solo parte statica
	    \item inner class: anche enclosing class, accessibile ovunque dall'outer
	    \item local class: anche argomenti/variabili, accessibile da un solo metodo
	    \item anonymous class: per creare un oggetto, senza un nuovo costruttore
	}}
    }}
}

\fr{Preview lambda expressions}{
    \bl{Un pattern molto ricorrente}{\iz{
	\item Avere classi anonime usate per incapsulare metodi ``funzionali''
	\item Java 8 introduce le lambda come notazione semplificata
	\item È il punto di partenza per la combinazione OO + funzionale
    }}
    \sizedrangedcode{\scriptsize}{5}{100}{\ecl/nested/UseSortLambda.java}
}


\section{Enumerazioni}

\frs{5}{Enumerazioni}{
  \bl{Motivazioni}{\iz{
    \item in alcune situazioni occorre definire dei tipi che possono assumere solo un numero fissato e limitato di possibili valori, che non cambierà in futuro
    \item Esempi:{\iz{
	\item le cifre da 0 a 9, le regioni d'Italia, il sesso di un individuo, i 6 pezzi negli scacchi, i giorni della settimana, le tipologie di camere di un hotel, le scuole di un ateneo, eccetera
    }}
  }}
  \bl{Possibili realizzazioni in Java}{\iz{
    \item usare delle \cil{String} rappresentando il loro nome: astrazione errata
    \item usare degli \cil{int} per codificarli (come in C): di basso livello
    \item usare una classe (e singolo oggetto) per elemento: prolisso
  }}
  \bl{Enumerazioni: \cil{enum \{ ...\}}}{\iz{
    \item consentono di elencare i valori, associando ad ognuno un nome
    \item è possibile collegare metodi e campi ad ogni ``valore''
  }}
}

\fr{Esempio classe \Cil{Person}}{
  \sizedrangedcode{\ssmall}{3}{100}{\ecl/enums/strings/Person.java}
}

%\fr{Esempio classe \Cil{Person}: 2/2}{
%  \sizedrangedcode{\scriptsize}{19}{100}{\ecl/enums/strings/Person.java}
%}

\fr{\Cil{UsePerson}}{
  \sizedrangedcode{\scriptsize}{3}{100}{\ecl/enums/strings/UsePerson.java}
}

\fr{Soluzione alternativa (con int), \Cil{Person}}{
  \sizedrangedcode{\tiny}{3}{100}{\ecl/enums/ints/Person.java}
}


\fr{Soluzione alternativa (con int), \Cil{UsePerson}}{
  \sizedrangedcode{\ssmall}{3}{100}{\ecl/enums/ints/UsePerson.java}
}

\fr{Soluzione con polimorfismo}{
  \sizedrangedcode{\ssmall}{3}{27}{\ecl/enums/polymorphism/UseRegionWithPolymorphism.java}
}

\fr{Soluzione con polimorfismo}{
  \sizedrangedcode{\ssmall}{28}{100}{\ecl/enums/polymorphism/UseRegionWithPolymorphism.java}
}


\frs{5}{Discussione}{
    \bl{Approccio a stringhe}{\iz{
	\item Penalizza molto le performance spazio-tempo
	\item Può comportare errori gravi per scorrette digitazioni
	\item Difficile intercettare gli errori
    }}
    \bl{Approccio a interi -- soluzione pre-enumerazioni}{\iz{
	\item Buone performance ma cattiva leggibilità
	\item Può comportare comunque errori, anche se più difficilmente
	\item L'uso delle costante è un poco dispersivo
    }}
    \bl{Approccio a polimorfismo: uso di classi diverse per ogni valore}{\iz{
	\item Difficilmente praticabile con un numero molto elevato di valori
	\item Prolisso, anche in termine di generazione di oggetti
	\item Tuttavia è sicuro e estendibile!{\iz{
	  \item Previene gli errori che si possono commettere
	  \item Consente facilmente di aggiungere nuovi elementi
	 }}
    }}
}

\fr{Soluzione pre enumerazioni: costanti e costruttore privato}{
  \sizedrangedcode{\ssmall}{3}{100}{\ecl/enums/constants/Region.java}
}

\fr{\Cil{UseRegion}}{
  \sizedrangedcode{\ssmall}{3}{100}{\ecl/enums/constants/UseRegion.java}
}

\fr{\Cil{enum} in Java}{
    \bl{Un nuovo tipo di dato}{\iz{
	\item Simile ad una classe
	\item Realizza l'approccio a costanti e costruttore privato
	\item Ottime performance, l'oggetto è già disponibile
	\item Impedisce interamente errori di programmazione
	\item Il codice aggiuntivo da produrre non è elevato
    }}
    \bl{Unica precauzione}{\iz{
      \item Andrebbero usate per insiemi di valori che difficilmente cambieranno in futuro
      \item Difficile modificare il codice successivamente
    }}
}

\fr{\Cil{enum Region}}{
  \sizedrangedcode{\scriptsize}{3}{30}{\ecl/enums/enumeration/Region.java}
  \sizedrangedcode{\scriptsize}{3}{30}{\ecl/enums/enumeration/UseEnum.java}
}

\fr{\Cil{Person} con uso della \Cil{enum}}{
  \sizedrangedcode{\ssmall}{5}{100}{\ecl/enums/enumeration/Person.java}
}


\fr{\Cil{UsePerson} con uso della \Cil{enum}}{
  \sizedrangedcode{\ssmall}{3}{100}{\ecl/enums/enumeration/UsePerson.java}
}


\fr{Metodi di default per ogni \Cil{enum}}{
  \sizedrangedcode{\ssmall}{5}{100}{\ecl/enums/enumeration/UseRegion.java}
}



\fr{Metodi aggiuntivi nelle \Cil{enum}}{
  \sizedrangedcode{\tiny}{3}{100}{\ecl/enums/parameters/Region.java}
}

\fr{Meccanismi per le \Cil{enum}}{
    \bl{Riassunto}{\iz{
	\item Esistono metodi istanza e statici disponibili per \cil{Enum}
	\item Si possono aggiungere metodi
	\item Si possono aggiungere campi e costruttori
    }}
    \bl{Riguardando la \cil{enum Regione}}{\iz{
	\item \`E una classe standard, con l'indicazioni di alcuni oggetti predefiniti
	\item I 20 oggetti corrispondenti alle regioni italiane
    }}
    \bl{Quindi}{\iz{
	\item \`E possibile intuirne la realizzazione interna
	\item E quindi capire meglio quando e come usarli
	\item[$\Rightarrow$] In caso in cui i valori sono ``molti e sono noti'', oppure..
	\item[$\Rightarrow$] Anche se i valori sono pochi, ma senza aggiungere troppi altri metodi..
    }}
}

\fr{\Cil{enum} innestate}{
    \bl{Motivazione}{\iz{
	\item Anche le \cil{enum} (statiche) possono essere innestate in una classe o interfaccia o enum
	\item Questo è utile quando il loro uso è reputato essere confinato nel funzionamento della classe outer
	\item \dots oppure quando il concetto dipende da quello della classe outer
    }}
}

\end{document}

