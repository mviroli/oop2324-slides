\documentclass[presentation]{beamer}
\usepackage{oop-slides}

\title[OOP01: Object-orientation]{01 \\ Ingegneria e astrazione object-oriented}

\begin{document}

\frame[label=coverpage]{\titlepage}

\section{Sistemi software e Ingegneria}

\fr{Sistemi software}{
  \bl{Obbiettivo di questo corso}{
    Insegnare tecniche moderne per la costruzione di \alert{sistemi software}\\
    $\Rightarrow$ utilizzando i concetti di \alert{programmazione ad oggetti}, che sono ``mainstream''
  }
  \bl{Programma vs sistema software}{
    \alert{Programma}: un set di istruzioni che automatizzano la soluzione di una classe di problemi, spesso associato ad una visione algoritmica della computazione (Input $\rightarrow$ Output).\\
    \alert{Sistema software}: un'aggregazione di componenti di varia natura (programmi, librerie, parti del sistema operativo, basi di dati, interfacce grafiche, servizi Web, rete, dispositivi hardware) che cooperano per fornire una funzionalità computazionale.
  }
}

% \fr{Da algoritmi a sistemi}{
%   \bl{Esempio: Versioni di una applicazione Calcolatrice}{\iz{
%     \item Algoritmica (input/output da linea di comando -- comando Linux \texttt{bc})
%     \item Interattiva (comandi successivi e stato  -- comando Linux \texttt{dc})
%     \item Con GUI (interfaccia grafica -- calcolatrice del PC)
%     \item Web (client sul browser, server remoto -- \texttt{http://web2.0calc.com/})
%    }}
%    \bl{Altri aspetti (per altri tipi di sistemi)}{\iz{
%     \item Persistenza (salvataggio dati e espressioni su file)
%     \item Integrazione con H/W (app integrata in uno smartphone, che accede ai suoi sensori)
%   }}
% }

\fr{Casi di sistemi software}{
  \bl{Esempi}{\iz{
    \item Calcolatrice matematica
    \item Applicazione per gestire esami universitari
    \item Simulatore di circuiti elettronici, movimento di folle, movimento pianeti 
    \item Monitor e elaborazione di dati biometrici (pressione, dati cardiaci,..)
    \item App per smartphone che mostra dove si trovano i miei amici di FB
    \item Video-game in cui personaggi vivono in un ambiente virtuale
    \item Controllore per dispositivi domotici (luce, termosifoni, acqua)
  }}
  \bl{Caratteristiche comuni}{\iz{
    \item Non meri programmi, ma sistemi ``pilotati'' da software
    \item Includono componenti algoritmiche, ma anche interazioni complesse
    \item Alcuni realizzabili già alla fine di questo corso OOP, altri nei 5 anni
  }}
}

\fr{Fasi del processo di sviluppo del software}{
  \bl{Analisi}{
    Si definisce in modo preciso il \alert{problema} da risolvere (non la soluzione!)
  }
  \bl{Design}{
    Si definisce la struttura del sistema da sviluppare\iz{
    \item progetto architetturale + progetto di dettaglio
    \item si descrive la \alert{soluzione}, ad uno specifico livello di dettaglio
  }}
  \bl{Implementazione/codifica}{ 
   Si realizza il sistema sulla base del progetto, scegliendo le tecnologie adeguate (efficienti, efficaci) -- p.e. il linguaggio di programmazione 
  }
  \bl{Post-codifica: Collaudo, Manutenzione, Deployment}{
   Fasi necessarie, che spesso impiegano più del 70\% delle risorse complessive
}}

\fr{Il problema dello sviluppo di sistemi software}{
  \bx{\texttt{http://it.wikipedia.org/wiki/Software\_crisis}}
  \bl{I progetti software spesso falliscono!}{\iz{
    \item Tipicamente: tempi non rispettati...
    \item Più raramente: impossibilità a proseguire
    \item Spesso: i programmatori a lungo andare si deprimono
    }}
  \bl{Quali cause possono comportare il fallimento di un progetto SW?}{\iz{
    \item Inadeguata analisi (requisiti non compresi appieno)
    \item Inadeguata (o assente) progettazione
    \item Cattive pratiche di programmazione 
    \item Aspetti organizzativi nel team di sviluppo
    }}
}

\frs{5}{Problem space vs solution space: il ``buon progetto''}{
 \bl{Problem space (fase di analisi)}{
  L'insieme delle entità/relazioni/processi nel mondo ``reale'' sulla base delle quali si formula il problema che il sistema software deve risolvere
  \iz{
  \item per il gestionale per esami universitari: studente, corso, corso di laurea, appello, voto, iscrizione
  }
 }
 \bl{Solution space (fase di progetto e implementazione)}{
   Il corrispondente insieme di entità/relazioni/processi nel mondo ``artificiale'' che devono risolvere il problema (realizzate mediante i linguaggi e le tecnologie scelte e a fronte del progetto). 
   \iz{
  \item funzione per il calcolo della media
  \item struttura dati per rappresentare i dati di uno studente
  \item form per visualizzare gli esami di uno studente
  }
 }
 \bx{\bf{Un buon progetto mappa al meglio il problem space nel solution space}}
}

\fr{Concetti del solution/problem space: ``livello di astrazione''}{
  \bl{Definizione di astrazione (nell'informatica)}{\iz{
     \item \`E il metodo usato per descrivere la parte importante di un sistema informatico complesso con lo scopo di facilitarne la progettazione, implementazione e manutenzione: si basa dul mettere da parte alcune caratteristiche specifiche ritenute non essenziali, concenrtandosi su quelle invece cruciali.
     \item Un \alert{livello di astrazione} è un insieme di concetti utilizzati per selezionare e descrivere in modo conveniente il sistema in oggetto
  }}
}

\fr{Il livello d'astrazione dei linguaggi di programmazione}{
  \bl{Astrazione e programmazione}{\iz{
    \item Ogni linguaggio di programmazione introduce un livello di astrazione
    \item I concetti del sistema da realizzare (nel gestionale esami: corsi, voti, studenti, calcolo media voti) devono essere tradotti nei ``costrutti'' forniti dal linguaggio
    \item \`E facile? \`E conveniente? \`E flessibile? Dipende dal linguaggio..
  }}
  \bl{Il livello di astrazione del C (e della progr. procedurale/imperativa)}{\iz{
    \item Stato del sistema | \`E costituito da strutture dati (costruite con tipi primitivi, array, puntatori e struct) tenute in stack e/o heap
    \item Dinamica | Esecuzione di procedure imperative, che si richiamano l'un l'altra
    \item Organizzazione | Librerie come set di funzioni, ricongiunte in un unico programma all'atto della compilazione/linking
  }}
}

\fr{I limiti del linguaggio C}{
  \bl{Il C porta ad una visione piuttosto machine-oriented}{\iz{
    \item \`E un livello di astrazione fortemente influenzato dall'HW sul quale si eseguono i programmi (CPU, memoria)
    \item Ciò porta a varie inadeguatezze
    {\iz{
      \item uso criptico di direttive/parametri di compilazione
      \item allocazione/deallocazione dinamica della memoria via librerie
      \item difficile controllo degli errori di esecuzione
      \item difficoltà a controllare gli aspetti HW-dependent
      \item difficoltà a modificare codice già ``acquisito''
    }}
    \item Nota: il C nasce negli anni '70 per rimpiazzare l'Assembly nell'implementazione del sistema operativo UNIX
  }}
  \bl{La direzione dei linguaggi moderni -- o di ``alto livello'' (d'astrazione)}{\iz{
    \item Introdurre un livello di astrazione vicino al problema da risolvere, ignorando il più possibile i dettagli dell'HW per risolverlo
  }}
}

\frs{5}{L'ecosistema dei linguaggi di programmazione}{
  \bl{Linguaggi e livelli di astrazione / paradigma}{\iz{
    \item C, Pascal: Computing function/procedure over data structures
    \item Lisp,ML: Everything is a function
    \item Prolog: Everything is a decision predicate
    \item Java,C++,C\#: Everything is an object (OO Programming)
    \item[$\Rightarrow$] L'OOP si è dimostrata ideale per sistemi complessi general-purpose
  }}
  \bl{L'evoluzione del ``mainstream''}{
    Machine Lang $\xrightarrow{'50-'60}$ Assembly $\xrightarrow{'70-'80}$ C $\xrightarrow{'90-2000}$ OOP (Java,..) $\xrightarrow{?}$ ?
  }
  \bl{Il futuro dei linguaggi (e anche la direzione dei linguaggi OOP)}{\iz{
    \item OO + funzionale + concorrenza: Scala, Kotlin, Java 8+
  }}
  \bl{Il caso dei linguaggi dinamici...}{\iz{
    \item Python, Ruby, JavaScript: orientati a scripting e piccola-media/scala
  }}
}

\fr{I vantaggi della programmazione ad oggetti}{
  \bl{Vantaggi}{\iz{
    \item Poche astrazioni chiave (classe, oggetto, metodo, campo)
    \item Usabili sia in progettazione che in codifica
    \item Supporto a estendibilità e riuso
    \item Supporto alla costruzione di librerie di qualità
    \item Facilmente integrabile in linguaggi C-like
    \item Eseguibile con alta efficienza
    \item[$\Rightarrow$] (quasi) tutti aspetti da sviscerare nel corso
}}
  \bl{Le critiche all'OOP}{\iz{
    \item Serve molta disciplina per ``scalare bene'' con la complessità del problema, ossia, per non incorrere in problemi di gestibilità al crescere della complessità del sistema da realizzare
    \item Altri paradigmi (p.e., funzionale) suggeriscono in parte come farlo...
}}
}

\section{L'astrazione object-oriented}

\fr{L'astrazione object-oriented (OO)}{
  \bx{
  \Huge Un oggetto ha stato, \\ comportamento e identità.
  }
}

\fr{Definizione più dettagliata}{
  \bx{\en{
   \item {\bf Everything is an object.} Un oggetto è una entità che fornisce operazioni per essere manipolata.
   \item {\bf Un programma è un set di oggetti che si comunicano cosa fare scambiandosi messaggi.} Questi messaggi sono richieste per eseguire le operazioni fornite.
   \item  {\bf Un oggetto ha una memoria fatta di altri oggetti.} Un oggetto è ottenuto impacchettando altri oggetti.
   \item {\bf Ogni oggetto è istanza di una classe.} Una classe descrive il comportamento dei suoi oggetti.
   \item {\bf Tutti gli oggetti di una classe possono ricevere gli stessi messaggi.} La classe indica tra le altre cose quali operazioni sono fornite, quindi per comunicare con un oggetto basta sapere qual è la sua classe.
  }}
}

\fr{Oggetto e sistema ad oggetti}{
  \fg{width=0.9\textwidth}{01/img/objects.pdf}
}


\fr{Sono questi concetti utili per il problem space?}{
    \bl{Esempio di sistema reale: la gestione informativa di un ateneo}{\iz{
	\item Come è opportuno organizzarla? Quale servizi fornisce?
    }}
    \bl{Visione object-oriented}{\iz{
	\item Un corso di laurea (CdL) è un ``oggetto''
	\item Che operazioni consente? Iscrivi studente, laurea studente, assegna docente a corso, concludi anno accademico,..
	\item Un programma (sistema) è fatto da oggetti: facoltà, corso di laurea, corsi, studente, docente
	\item Le interazioni fra questi oggetti sono ``scambi di messaggi''
	\item Il CdL ha uno stato fatto da altri oggetti: docenti, studenti, corsi
	\item Il CdL è istanza di una classe: la classe dei CdL (con una gestione comune) -- UNIBO ne gestisce dozzine contemporaneamente
    }}
}

\fr{OO: problem space, solution space}{
    \bl{L'esperienza mostra che:}{\iz{
	\item \`E piuttosto agevole modellare sistemi reali (o artificiali) come sistemi orientati agli oggetti
	\item Infatti, gli strumenti di modellazione standard usano il paradigma object-oriented! (Vedi UML)
    }}
    \bl{Il vantaggio delle soluzioni object-oriented}{\iz{
	\item Consentono di ``portare'' il problem space nel solution space in modo diretto
	\item Usando gli stessi concetti object-oriented anche a livello di programmazione
	\item ..ossia supportano il concetto di ``buon progetto'' che abbiamo discusso
    }}
}


\fr{Overview}{
\bx{\en{
\item Ogni oggetto ha una interfaccia
\item Un oggetto fornisce un servizio
\item Un oggetto deve nascondere l'implementazione
\item Le implementazioni possono essere riusate
\item Il riuso tramite ereditarietà
%\item Polimorfismo e sostituibilità
%\item Le gerarchie di classi a radice singola
%\item I Container
}}
}

\fr{Ogni oggetto ha una interfaccia}{
  \bl{Classi, istanze, metodi, interfaccia}{\iz{
    \item Oggetti simili sono istanze della stessa \alert{classe}, o \alert{tipo}
    \item La classe definisce i messaggi ricevibili, attraverso \alert{metodi}
    \item L'insieme dei metodi prende il nome di \alert{interfaccia}
    \item Un messaggio ha effetto su stato e comportamento dell'oggetto
  }}
  \bx{Esempio ``lampadina'' in notazione UML (Unified Modelling Language):}
  \fg{width=0.6\textwidth}{01/img/light.png}
}

\fr{Un oggetto fornisce un servizio}{
  \bl{Quale entità del problem space deve diventare un oggetto?}{\iz{
    \item Conviene considerare un oggetto come un fornitore di un servizo
    \item Tutto il programma può essere visto come un servizio dato all'utente
    \item Principio di decomposizione: i sotto-servizi sono affidati ai vari oggetti
  }}
  \bl{Vantaggi di questo approccio}{\iz{
    \item Semplifica la progettazione degli oggetti, e il mapping col problema
    \item Semplifica il loro riuso in programmi diversi -- come parte di libreria
    \item Semplifica la comprensione dei programmi, specialmente da terzi
  }}
  \bx{{Opportunità: $\Rightarrow$ Se non se ne riesce a descrivere il servizio, allora probabilmente un oggetto non ha ragione d'esistere!}\\
      {Coerenza: $\Rightarrow$ Se sembra che un oggetto realizzi due servizi diversi, allora probabilmente bisogna in realtà realizzare due oggetti!}}
}

\fr{Un oggetto deve nascondere l'implementazione}{
  \bl{Le due figure: creatore di classi vs. programmatore cliente}{
  \iz{
    \item Chi produce la classe (e ha la responsabilità del suo funzionamento)
    \item Chi usa la classe (per fornire un servizio più di alto livello)
  }}
  \bl{Information hiding}{\iz{
  \item Il creatore rende visibile solo una piccola parte della classe
  \item Il resto è invisibile perché suscettibile di modifiche future
  \item Principio: ``less is more''
  \item Tipica struttura di una classe \iz{
   \item interfaccia: la sola visibile al cliente, di norma
   \item \alert{membri} (o \alert{campi}) della classe (i sotto-oggetti di cui è costituito)
   \item implementazione metodi (cosa fa l'oggetto quando riceve messaggi){\iz{
    \item come cambia lo stato (membri)
    \item quali messaggi manda ad altri oggetti
    \item quale risultato fornisce (risposta al messaggio)
   }}
   
  }
  }}
}

\fr{Le implementazioni possono essere riusate}{
  \bl{Fasi del processo di riuso}{
  \en{
    \item Il creatore produce una classe e ne verifica il corretto funzionamento: diventa una unità riusabile di codice
    \item Un cliente può ri-utilizzarla per creare nuovi concetti
  }}
  \bl{La tecnica di riuso più usata è la \alert{composizione}}{\iz{
  \item Un nuovo oggetto è costituito da oggetti di altre classi
  \item Relazione chiamata ``has-a'' (``ha un'')
  \item Questa relazione può essere nascosta, e resa dinamica
  }}
  \bx{Esempio ``car+engine'' in notazione UML:}
  \fg{width=0.6\textwidth}{01/img/car.png}
}

\frs{5}{Il riuso tramite ereditarietà}{
  \bl{\`E una ulteriore, fondamentale, tecnica di riuso}{\iz{
  \item Un nuovo oggetto(/classe) estende i servizi di uno esistente
  \item Fornisce i metodi della sopra-classe, ma anche altri
  \item Relazione chiamata ``is-a'' (``è un'')
  }}
  \bx{Esempio ``shape'' in notazione UML:}
  \fg{width=0.4\textwidth}{01/img/shape.png}
}

% \fr{Polimorfismo e sostituibilità}{
%   \bl{La conseguenza cruciale dell'ereditarietà}{\iz{
%   \item Un nuovo oggetto di tipo \cil{Triangle} fornisce tutti i servizi di \cil{Shape}
%   \item Quindi è sempre possibile sostituire (dinamicamente) ad un oggetto di tipo \cil{Shape} un oggetto di tipo \cil{Triangle}, perché quest'ultimo ne capirà tutti i messaggi e svolgerà gli stessi compiti, ma in modo specializzato.
%   \item Ad esempio, si può pensare ad una figura come ad una collezione di \cil{Shape} disegnabili dall'utente, che dinamicamente potranno essere \cil{Triangle}, \cil{Circle}, \cil{Square}, eccetera.
%   }}
%   \bl{Polimorfismo = molte forme}{\iz{
%   \item \`E una tecnica importante indipendentemente dall'OO
%   \item Consente ad un tipo di dato di assumere forme diverse
%   \item E quindi essere utilizzabile in vari contesti
%   }}
% }
% 
% \fr{Le gerarchie di classi a radice singola}{
%   \bl{Le gerarchie di classi}{\iz{
%   \item L'ereditarietà spinge a costruire gerarchie (alberi) di classi
%   \item ..per effetto di processi di classificazione delle entità
%   }}
%   \bl{La radice: \cil{Object}}{\iz{
%   \item Molti linguaggi OO (come Java e C\#, non il C++) hanno una classe speciale da cui tutte ereditano, solitamente chiamata \cil{Object}
%   \item Questo comporta numerosi vantaggi, fra cui la possibilità di dotarla di servizi generali che tutte le classi quindi offronto (stampa a video, garbage collection, gestione errori)
%   \item Un altro vantaggio è la possibilità di costruire ``container'' generali
%   }}
% }
% 
% \fr{I Container}{
%   \bl{Strutture dati e OO}{\iz{
%   \item La soluzione di certi problemi può essere costruita con un numero imprecisato di oggetti dello stesso tipo (ma anche di tipo diverso)
%   \item p.e., archivi, insiemi, aggregati dinamici
%   \item Bisogna predisporre delle strutture (oggetti) container che siano in grado di organizzarli in modo efficace ed efficiente
%   \item Spesso realizzati usando algoritmi e strutture dati note
%   }}
%   \bl{Container e Librerie}{\iz{
%   \item Una parte fondamentale delle librerie dei linguaggi OO è quella che fornisce dei container flessibili e performanti
%   \item Implementano strutture note: \cil{Array}, \cil{ArrayList}, \cil{LinkedList}, \cil{TreeSet}, \cil{HashMap},..
%   }}
% }

\fr{Altri aspetti -- organizzazione del corso}{
  \bl{Funzionalità base -- I parte}{\iz{
    \item I concetti appena introdotti
  }}
  \bl{Funzionalità aggiuntive -- II parte}{\iz{
  \item Polimorfismo per genericità
  \item Gestione delle eccezioni
  \item Riflessività, classi innestate, enumerazioni
  }}
  \bl{Tecniche Avanzate -- III parte}{\iz{
  \item Librerie avanzate (I/O, GUI, Concorrenza)
  \item Integrazione con la programmazione funzionale
  \item Pattern di progettazione
  \item Prassi di programmazione efficace
  %\item Altre piattaforme (Android, C\#)
  }}
}

\section{Java}

\frs{5}{Qualche nota introduttiva..}{
  \bl{Java}{\iz{
    \item Linguaggio inventato da J. Gosling, alla Sun Microsystems (1995)
    \item \dots ora è gestito commercialmente da Oracle, con anche versioni ``open'' che sono comunque ``reference'' (OpenJDK)
    \item Una semplificazione del C++ (ossia C + OO), per sistemi embedded
    \item Filosofie: ``Write once run everywhere'' + ``Keep it simple, stupid''{\iz{
      \item \dots la prima filosofia porta all'uso di Java sia su PC, che sistemi embedded (Raspberry), o mobile (Android)
      \item \dots la seconda filosofia necessariamente abbandonata nel tempo.
    }}
    \item Al momento il terzo linguaggio più popolare secondo l'indice Tiobe{\iz{
	\item fonte: \myurl{https://www.google.it/search?q=tiobe+tcpi}
		\item (indice usato per decisione strategiche nello sviluppo di nuovi SW)
    }}
    \item \dots C, Java e Python al momento quasi equivalenti
  }}
  \bl{In questo corso}{\iz{
    \item Useremo Java come riferimento per programmazione/progettazione
    \item Le guideline progettuali che daremo sono di validità generale
    \item Forniremo anche gli elementi base del linguaggio C\#
  }}
}

\fr{Strumenti Java}{
  \bl{Java Development Kit (JDK) -- useremo la versione OpenJDK 11}{\iz{
    \item Insieme di tool per lo sviluppatore
    \item \myurl{https://openjdk.java.net/} o meglio \myurl{https://adoptopenjdk.net/}
  }}
  \bl{Java Virtual Machine (JVM) -- inclusa nel JDK}{\iz{
    \item Un programma (C/C++) che carica i binari delle classi e le esegue
    \item Fornisce indipendenza dall'HW, e servizi aggiuntivi (SISOP, GC)
  }}
  \bl{Schema compilazione/interpretazione}{\en{
    \item Compilazione dei programmi col compilatore Java (comando \texttt{javac})
    \item Esecuzione del programma con la JVM (comando \texttt{java})
  }}
}

\fr{Write once run everywhere}{
\fg{height=0.8\textheight}{01/img/java-program-execution.png}
}

\frs{5}{Sviluppo storico di Java}{
\bl{Pricipali versioni}{\iz{
  \item Java (1996): versione base iniziale
  \item Java 2 (1998): aggiunta del framework Swing
  \item Java 5 (2004): aggiunta di generici, inner class, annotazioni
  \item Java 8 (2014): aggiunta lambda expression e streams
  \item Java 9 (2017): aggiunta Java Module System
  \item Java 10 (2018): aggiunta local variable type-inference
  \item[$\Rightarrow$] Java 11 (2018): versione ``long-term support'' (LTS) che useremo
  \item Java 12$\rightarrow$16 (2019-2021): piccole aggiunte
  \item Java 17 (sept. 2021): LTS appena giunta, non la useremo
}}
\bl{Modello di sviluppo attuale}{\iz{
  \item Da Java 9, si ha una nuova release ogni 6 mesi
  \item Java 11 è (quasi) l'ultima LTS, e si raccomanda di non rimanere indietro
  \item Oracle detiene il marchio e una implementazione ``commerciale''
  \item L'iniziativa OpenJDK produce una ulteriore implementazione open che è di riferimento -- quella che useremo noi
  \item Esistono anche altre JVM (Eclipse OpenJ9, Amazon Corretto,...)
}}
}


%\fr{Motto della lezione}


\end{document}
