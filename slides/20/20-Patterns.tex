\documentclass[presentation]{beamer}
\usepackage{../oop-slides}

\title[OOP20: OO Design and Patterns]{20 \\ Progettazione OO e pattern}

\begin{document}


\newcommand{\ecl}{\idepath{p20patterns}}
\newcommand{\ecllambda}{\idepath{p17lambda}}

\frame[label=coverpage]{\titlepage}

\fr{Outline}{
  \bl{Goal della lezione}{\iz{
  \item Illustrare il ciclo di vita del software
  \item Illustrare il concetto di Design Pattern
  \item Discutere i dettagli di vari Pattern
  \item Ripassare vari concetti del corso
  }}
  \bl{Argomenti}{\iz{
    \item Progettazione architetturale
    \item La nozione di Pattern
    \item La lista di Pattern della GoF
    \item Dettagli su Pattern già visti
    \item Dettagli su nuovi Pattern
  }}
}

\section{Ciclo di vita del software}

\frs{10}{Macro-fasi del processo di sviluppo}{
  \bl{Analisi}{
    Si definisce in modo preciso il \alert{problema} da risolvere{\iz{
      \item i requisiti del problema, ben ``ingegnerizzati''
      \item il modello del dominio (terminologia, entità e relazioni)
    }}
  }
  \bl{Design}{
    Si definisce la struttura del sistema da sviluppare\iz{
    \item progetto architetturale + progetto di dettaglio
    \item si descrive la \alert{soluzione}, ad uno specifico livello di dettaglio
  }}
  \bl{Implementazione/codifica}{ 
   Si realizza il sistema sulla base del progetto, scegliendo le tecnologie adeguate (efficienti, efficaci) -- p.e. il linguaggio di programmazione 
  }
  \bl{Post-codifica: Collaudo, Manutenzione, Deployment}{
   Fasi necessarie, che spesso impiegano più del 70\% delle risorse complessive
}}

\frs{10}{Micro-fasi in analisi e progettazione}{
  \bl{Analisi -- interazione col committente}{
    I requisiti raccolti dalle interazioni col committente vengono analizzati e formalizzati, diventando una sorta di ``contratto'' di come il software deve apparire e funzionare (non di come deve essere internamente realizzato). Si potrebbero aggiungere studi di fattibilità per consentire di comprendere la complessità del problema.
  }
  \bl{Design architetturale -- inizio della progettazione}{
    Progettazione ``ad alto livello'', in cui si definisce solo la struttura complessiva del sistema in termini dei principali moduli (classi, o meglio interfacce) di cui esso è composto e delle relazioni macroscopiche (``uses'', ``has-a'' o ``is-a'') fra di essi.
  }
  \bl{Design di dettaglio -- progettazione}{
    Progettazione a ``più basso livello'', che individua una organizzazione molto vicina alla codifica, ovvero che la vincola in maniera sostanziale. Descrive interfacce, classi astratte e concrete che rappresentano la \emph{soluzione} ai principali problemi identificati in analisi
  }
}

\fr{Sul modello di sviluppo}{
\bl{Quale approccio?}{\iz{
  \item A cascata: le fasi rigorosamente in ordine temporale
  \item A spirale: le fasi rigorosamente in ordine ma svolte in più ``cicli''
  \item A fontana: si può tornare temporaneamente nella fase precedente/successiva
  \item Agile: sviluppo iterativo e incrementale
  \item[$\Rightarrow$] non esiste il modello ``perfetto''
}}
\bl{Note e consigli per piccoli progetti (come quello d'esame)}{\iz{
  \item Conviene darsi 2-3 obbiettivi intermedi incrementali
  \item Ogni obbiettivo usi un approccio simile a quello in cascata
  \item La documentazione finale campioni il sistema finale, simulando un modello a cascata: ciò consente una coerente analisi a posteriore del sistema costruito
}}
}

\section{Progettazione architetturale}

\frs{5}{Progettazione architetturale}{
    \bl{Idee chiave}{\iz{
      \item Costruire un unico diagramma dei concetti principali (classi/interfacce)
      \item Idealmente, meglio se fatto di sole interfacce, indicativamente 5-12 in tutto 
      \item Non ci siano classi/interfacce isolate dal resto
      \item Il sistema va diviso in componenti il più possibile isolati e non dipendenti fra loro
      \item Una descrizione in prosa indichi il principale ruolo di ogni entità e le relazioni esistenti -- questo vale per ogni diagramma UML! 
      \item Meglio se non estratto dal codice, ma scritto ``a mano''
    }}
    \bl{Quale ``pattern'' architetturale?}{\iz{
      \item MVC è un ottimo punto di partenza
      \item Non l'unico, ma non ne vedremo altri
      \item Attenzione a ``copiare'' architetture/progetti dalla rete!{\iz{
         \item per OOP: meglio partire dal foglio bianco
      }}
    }}
}


\fr{Il caso di MVC -- schema generale (nomi di comodo)}{
    \fg{height=0.55\textheight}{img/MVC-abstract.png}
}

\fr{Linee guida per un buon MVC pt.1}{
    \bl{Ruoli/responsabilità delle 3 parti}{\iz{
	\item View: gestire la parte di presentazione e di interazione con l'utente
	\item Model: gestire i dati e la logica di dominio dell'applicazione
	\item Control: gestire la ``meccanica'' dell'applicazione, coordinando View, Model ed eventuali interazioni con altri componenti (SISOP)
    }}
    \bl{Interazioni}{\iz{
      \item Le interazioni dell'utente in V diventano eventi catturati da C, o metodi \cil{void} chiamati su C
      \item C osserva / modifica / o interroga M
      \item Quando necessario C comanda modifiche a V
      \item[$\Rightarrow$] non vi sono altre interazioni possibili!
    }}
}

\fr{Linee guida per un buon MVC pt.2}{
    \bl{Linee guida}{\iz{
      \item Cambiare la view e la sua tecnologia (Swing, AWT, Console) non deve far toccare M e C
      \item Il modello gestisce in modo elegante e ben costruito la logica di dominio dell'applicazione{\iz{
        \item In M c'è margine per fare dell'ottima e pulita progettazione OO
	\item Qualsiasi comportamento e/o fenomeno possibile è riproducibile da una opportuna sequenza di chiamate di metodo su M fatte da un \cil{main}
	\item Di fatto, C sostituisce questo \cil{main}, creando l'applicazione indipendentemente dai dettagli della UI
      }}
      \item Modello e vista espongono interfacce ben progettate (senza dettagli implementativi) attraverso cui C agisce in modo ben disaccoppiato
      \item Se esistono Thread o accessi al sistema operativo, sono incapsulati in C
      \item M, V e C potrebbero essere fatti da ``sotto-parti'' diverse
    }}
}



\section{Design Pattern e Progettazione di Dettaglio}

\fr{Progettazione di dettaglio}{
\bl{Elementi}{\iz{
\item Non descrive ogni singola classe/interfaccia del sistema
\item Descrive relazioni fra oggetti, quelle ritenute più importanti per capire come il sistema è organizzato
\item Quelle che nascondono elementi non banali
\item Documentata da più diagrammi UML sempre di 5-10 classi ognuno, sempre corredati dalla prosa necessaria
}}
\bl{Come progettare una buona classe o gruppo di classi?}{\en{
  \item buona conoscenza della programmazione OO e delle linee guida di buona programmazione/progettazione note e discusse
  \item utilizzo di cataloghi noti di pattern di progettazione (design pattern)
}}
}

\fr{Design Pattern}{
  \bl{I Pattern di progettazione}{\iz{
    \item Idea: trasmettere esperienze (positive) e ore di lavoro (di identificazione, rifattorizzazione) ad altri per essere usate \emph{tout court}
    \item Sono elementi riusabili (semplici ed eleganti) di progettazione OO
    \item Sono stati ottenuti in passato (e tuttora) dall'analisi di soluzioni ricorrenti in progetti diversi
    \item Alcuni sono particolarmente famosi, come quelli della ``Gang of Four'' (detti anche Pattern della GoF, o Pattern di Gamma){\iz{
	\item Testo famosissimo (in C++): ``Design Patterns: Elements of Reusable Object-Oriented Software'' di E.Gamma, R.Helm, R.Johnson,J.Vlissides
	\item 23 in tutto. Esempi: Decorator, Singleton, Template Method, Observer
	\item (Cit. ``SW di grosse dimensioni li usano praticamente tutti'')
    }}
    \item Il loro uso migliora molto il codice{\iz{
	\item Ne favorisce la comprensione se li si indicano nella documentazione
	\item Rende il codice più flessibile (nascono per questo)
	\item Portano più direttamente ad una buona organizzazione
    }}
  }}
}

\fr{Design Pattern in questa sede}{
  \bl{In questa lezione}{\iz{
    \item Illustreremo il catalogo della GoF
    \item Verranno approfonditi alcuni Pattern anche con esempi
  }}
  \bl{Nel corso}{\iz{
    \item Vari Pattern già stati utilizzati (es.: nelle librerie)
    \item Vanno usati dove opportuno nel progetto d'esame (e nella relazione)
    \item Possono essere tema dell'esame in laboratorio
    \item Quelli visti a lezione sono da conoscere tassativamente
  }}
  \bl{Per il vostro futuro}{\iz{
    \item Noi porremo le basi per uno loro studio in autonomia
    \item Un ottimo progettista li conosce e usa (ove opportuno) \alert{tutti}
    \item Alla magistrale verranno approfonditi
  }}
}

\frs{15}{Motivazioni}{
    \bl{Rifattorizzazione (refactoring)}{\iz{
      \item Operazione di modifica del codice che non aggiunge funzionalità
      \item Ha lo scopo di migliorare programmazione e struttura del SW
      \item Ha lo scopo di attrezzare il codice a possibili cambiamenti futuri
      \item Può/deve quindi comportare una riprogettazione di alcune parti
    }}
    \bl{La necessità del refactoring}{\iz{
	\item Una buona progettazione non la si ottiene al primo ``colpo'', ma richiede vari refactoring
	\item Brian Foote identifica tre fasi nello sviluppo di un sistem: prototyping, expansionary, consolidating; nel consolidamento si rifattorizza
	\item Nell'agile programming, ogni ciclo di sviluppo non parte se non si è rifattorizzato il codice del ciclo precedente (sia in cicli corti che lunghi)
    }}
    \bl{I pattern}{\iz{
	\item L'esperienza pregressa risulta fondamentale per velocizzare il processo di rifattorizzazione
	\item I Pattern di progettazione forniscono direttamente ``ricette'' di buona costruzione o rifattorizzazione del SW
    }}
}

\frs{5}{Principi di buona programmazione/progettazione OO}{
    \bl{DRY: Don't repeat yourself}{
        Never let two pieces of code be the same or even very similar in your code base
    }
    \bl{KISS: Keep it simple, stupid}{
        Always favour the simplest solution if adequate, understandability is key!
    }
    \bl{SRP: Single Responsibility Principle}{
        Each component (class) should have one reason to change
    }
    \bl{OCP: Open-closed principle}{
        Each component (class) should be closed to change but open to extension
    }
    \bl{DIP: Dependency-inversion principle}{
        Do not depend on others' implementations, abstract them into interfaces!
    }
}


\fr{Struttura}{
    \bl{Un Pattern ha quattro elementi fondamentali}{\iz{
	\item Un nome. (È un aspetto fondamentale!)
	\item Un problema che risolve. (La causa che porta al suo uso)
	\item La soluzione che propone. (Gli elementi del progetto)
	\item La conseguenza che porta. (Riuso, variabilità, performance,..)
    }}
    \bl{Granularità}{\iz{
	\item Gruppo ristretto di (1-5) oggetti/classi generali dipendenti fra loro
	\item Sistemi più specifici o più complessi sono utili, ma non propriamente dei ``Pattern''\iz{
	    \item Non singole classi riusabili (liste, hash-table)
	    \item Non ``pattern architetturali'' (come MVC)
	    \item Non framework complessi (gerarchia Swing, Reflection)
	}
    }}
}



%\fr{Classificazione dei Pattern}{
%    \fg{height=0.85\textheight}{img/classification.png}
%}

\fr{Classificazione dei Pattern: categorie}{
    \bl{Livello ``proposito del Pattern''}{\iz{
	\item Creazionali: Riguardano la creazione degli oggetti
	\item Strutturali: Riguardano la composizione di classi/oggetti
	\item Comportamentali: Riguardano la interazione e distribuzione di responsabilità fra classi/oggetti
    }}
    \bl{Livello ``scope''}{\iz{
	\item Classi: Il Pattern riguarda primariamente le relazioni fra classi (e sottoclassi), e quindi tratta aspetti statici (compile-time)
	\item Oggetti: Il Pattern riguarda primariamente le relazioni fra oggetti (l'esistenza di riferimenti fra oggetti), e quindi tratta aspetti dinamici (run-time)
    }}
}

\fr{I 23 Pattern GoF}{
    \bl{Creazionali}{\iz{
	\item A livello di classe: \alert{Factory Method}
	\item A livello di oggetto: \alert{Abstract Factory}, \alert{Builder}, Prototype, \alert{Singleton}
    }}
    \bl{Strutturali}{\iz{
	\item A livello di classe: \alert{Adapter}
	\item A livello di oggetto: \alert{Adapter}, Bridge, Composite, \alert{Decorator}, Facade, Proxy
    }}
    \bl{Comportamentali}{\iz{
	\item A livello di classe: Interpreter, \alert{Template Method}
	\item A livello di oggetto: Chain of Responsibility, Command, \alert{Iterator}, Mediator, Memento, Flyweight, \alert{Observer}, State, \alert{Strategy}, \alert{Visitor}
    }}
}

\fr{``Design for change''}{
    \bl{Problemi nel cercare di realizzare modifiche ad un sistema..}{\iz{
	\item Dipendenza dal nome di classe concreta{\iz{
	    \item Abstract Factory, Factory Method, Prototype
	}}
	\item Dipendenza da operazioni (metodi) specifici{\iz{
	    \item Chain of Responsibility, Command
	}}
	\item Dipendenza dalla interfaccia/implementazione di un oggetto{\iz{
	    \item Abstract Factory, Bridge, Memento, Proxy
	}}
	\item Dipendenza da un algoritmo specifico{\iz{
	    \item Builder, Iterator, Strategy, Template Method, Visitor
	}}
	\item Dipendenza stretta fra due classi{\iz{
	    \item Abstract Factory, Bridge, Chain of Responsibility, Command, Facade, Mediator, Observer
	}}
	\item Estendere funzionalità via subclassing non è pratico{\iz{
	    \item Bridge, Chain of Responsibility, Composite, Decorator, Observer, Strategy
	}}
    }}
}

\fr{Schema di descrizione per ogni pattern}{
    \bx{Aderiremo al seguente schema, che è una semplificazione di quello proposto alla GoF}
    \bl{Ingredienti}{\en{
	\item Descrizione in prosa (nome, motivazione, esempi, soluzione)
	\item Rappresentazione grafica (diagramma delle classi generale)
	\item Esempio (già visto/nuovo)
    }}
    \bl{Pattern già incontrati, alcuni da approfondire}{
	Singleton, Template Method, Strategy, Observer, Iterator, Decorator
    }
    \bl{Nuovi}{
	Factory Method, Abstract Factory
    }
}

\fr{I pattern nel corso OOP}{
    \bl{Esame laboratorio}{\iz{
	\item Qualche esercizio potrebbe riguardare il loro uso
    }}
    \bl{Esami di progetto}{\iz{
      \item Identificarne/usarne ``pochi'' è considerato poco soddisfacente
      \item Scegliere di usarli non è arbitario, ma è indice di buona progettazione e/o di buona rifattorizzazione
      \item Gli argomenti: ``in questo progetto non servivano'' e ``non c'è stato tempo'' sono pessimi
      \item Argomento di probabile discussione all'esame
    }}
    \bl{Di conseguenza}{\iz{
      \item I pattern qui presentati vanno conosciuti
      \item Gli altri pattern sono facoltativi, e importanti per il vostro futuro
    }}
}


\section{Pattern comportamentali/strutturali}

\frs{5}{Proxy: strutturale, su classi}{
    \bl{Intento/motivazione}{
	Fornire un surrogato (o placeholder) per un altro oggetto, per contrallarne l'accesso.
    }
    \bl{Esempi}{Controllare l'accesso ad un oggetto per:\iz{
	\item ottimizzare il costo della sua inizializzazione/use
	\item fornire accesso trasparente alla rete
	\item fornire limitazioni alla sua funzionalità (p.e., solo lettura)
    }}
    \bl{Soluzione}{\iz{
	\item Si crei un wrapper che delega l'oggetto da controllare, con medesima interfaccia e quindi da usare in modo del tutto trasparente
    }}
}


\fr{Proxy: UML}{
    \fg{height=0.5\textheight}{img/proxy.png}
}

\fr{Esempio: algoritmi come metodi statici}{
\sizedrangedcode{\ssmall}{3}{100}{\ecl/proxy/pre/Factorial.java}
\sizedrangedcode{\ssmall}{3}{100}{\ecl/proxy/pre/UseFactorial.java}
}

\fr{Soluzione inadeguata (viola SRP + OCP)}{
La gestione del caching è incastrata insieme a quella degli algoritmi, ed è statica
\sizedrangedcode{\ssmall}{6}{100}{\ecl/proxy/bad/Factorial.java}
}

\fr{Idea wrapper/proxy, con interfaccia unificante}{
\sizedrangedcode{\scriptsize}{3}{100}{\ecl/proxy/med/Factorial.java}
\sizedrangedcode{\scriptsize}{3}{100}{\ecl/proxy/med/FactorialImpl.java}
}

\fr{Gestore del caching}{
\sizedrangedcode{\ssmall}{9}{100}{\ecl/proxy/post/FactorialWithCache.java}
}

\fr{Proxy: conseguenze}{
    \bl{Effetti benefici}{\iz{
	\item Separa e incapsula la logica d'accesso
	\item Non sporca la classe che ralizza la funzionalità base
	\item Può essere reso trasparente al cliente
	\item Vari stadi di proxying possono essere usati in cascata e in alternativa
    }}
    \bl{I principi salvaguardati da Proxy}{\iz{
        \item SRP: una classe responsabile dell'algoritmo, una del caching
        \item OCP: l'aggiunta del caching è fatta con una nuova classe, non modificando l'algoritmo, e una nuova tecnica di caching è aggiungibile separatmente
        \item DIP: il client dipende dall'interfaccia, non dalla classe, e questo consente l'uso trasparente del proxy
    }}
}

\frs{15}{I metodi statici nella programmazione OOP}{
    \bl{Approccio del programmatore che non ha capito la OOP}{\iz{
        \item svariate classi con campi e metodi statici, in pratica programmando in C
    }}
    \bl{Approccio comunemente usato, ma sconsigliabile}{\iz{
        \item usati quando il metodo non usa informazioni del receiver
    }}
    \bl{Approccio più pulito, consigliato}{\iz{
        \item usati solo in classi con algoritmi di utilità, che hanno solo metodi statici
    }}
    \bl{Approccio moderno}{\iz{
        \item anche le classi di utilità abbiano interfaccia e implementazione
    }}
    \bl{E i campi statici?}{\iz{
        \item da usare solo per costanti: non abbiano mai side-effect
    }}
    \bl{Metodi statici frequentemente usati nei pattern}{\iz{
        \item Singleton: sempre meglio evitare di usarlo, è considerato anti-pattern
        \item Static factory: molto usato nelle librerie, sarebbe meglio usare Factory Method
    }}
}

\frs{5}{Singleton: creazionale, su oggetti}{
    \bl{Intento/motivazione}{
	Garantire che una classe abbia una unica istanza, accessibile globalmente e facilmente a molteplici classi, senza doversi preoccupare di fornirne il riferimento a chi lo richiede (ad esempio passandolo al costruttore)
    }
    \bl{Esempi}{\iz{
	\item Un unico gestore di stampanti in un sistema
	\item Un unico gestore del ``log''
	\item \cil{java.lang.Runtime}
    }}
    \bl{Soluzione}{\iz{
	\item La classe sia responsabile di tenere traccia di tale unica istanza
	\item La classe impedisca la creazione di altri oggetti
	\item La classe fornisca l'accesso a tale oggetto staticamente
	\item Attenzione: singleton accoppia clienti e implementazione
    }}
}

\fr{Singleton: UML}{
    \fg{height=0.5\textheight}{img/singleton.jpg}
}

\fr{Singleton: Il caso di \Cil{java.lang.Runtime}}{
    \sizedrangedcode{\scriptsize}{3}{100}{\ecl/singleton/UseRuntime.java}
}

\fr{Singleton: Il caso di una classe \Cil{Log}}{
    \sizedrangedcode{\scriptsize}{3}{100}{\ecl/singleton/Log.java}
    \sizedrangedcode{\scriptsize}{3}{100}{\ecl/singleton/UseLog.java}
}

\frs{5}{Singleton: conseguenze}{
    \bl{Effetti benefici}{\iz{
	\item C'è un controllo ``incapsulato'' di chi vi accede
	\item Evita di dover portare i riferimenti all'oggetto nei campi di tutti le classi che lo usano
	\item È facile raffinare l'implementazione del singleton (via subclassing)
	\item Può gestire la creazione by-need (detta anche \emph{lazy}) dell'oggetto
	\item Più flessibile dei metodi statici (che non hanno overriding)
    }}
    \bl{Critiche}{\iz{
	\item Il Singleton può essere problematico col multi-threading
	\item Crea dipendenze nascoste, gli user dipendono dal nome della classe
	\item Difficile tornare indietro dalla scelta di usare il singleton
	\item Incapsula due responsabilità distinte (creazione + aspetti interni)
	\item Rende meno estendibile il codice della classe (è meno ``OOP'')
	\item[$\Rightarrow$] \alert{Da usare solo quando: (i) è cruciale che ci sia un solo oggetto di quella classe, e (ii) far arrivare il riferimento all'oggetto a tutte le classi che ne necessitano sarebbe troppo complesso}
    }}
}

\fr{Singleton con ``lazy initialization'' (non thread-safe)}{
    \sizedrangedcode{\scriptsize}{3}{100}{\ecl/singleton/LogLazy.java}
}

\fr{Singleton con ``lazy initialization'' e ``thread-safe''}{
    \sizedrangedcode{\ssmall}{3}{100}{\ecl/singleton/LogLazyTS.java}
}


\frs{5}{Template Method: comportamentale, su classi}{
    \bl{Intento/motivazione}{
	Definisce lo scheletro (template) di un algoritmo (o comportamento), lasciando l'indicazione di alcuni suoi aspetti alle sottoclassi.
    }
    \bl{Esempi}{\iz{
	\item In un input stream (\cil{InputStream}), i vari metodi di lettura sono dei Template Method: dipendono dall'implementazione del solo concetto di lettura di un \cil{int}
	%\item Similmente, i metodi di \cil{AbstractSet} tranne \cil{size()} e \cil{iterator()}
	\item Le interfacce funzionali con metodi di default che chiamano l'astratto
    }}
    \bl{Soluzione}{\iz{
	\item L'algoritmo è realizzato attraverso un metodo non astratto (il template method) di una classe astratta
	\item Questo realizza l'algoritmo, chiamando metodi astratti quando servono gli aspetti non noti a priori
	\item Una sottoclasse fornisce l'implementazione dei metodi astratti
    }}
}

\fr{Template Method: UML}{
    \fg{height=0.5\textheight}{img/template_method.jpg}
}

\fr{Template Method: Una estensione di \Cil{InputStream}}{
    \sizedrangedcode{\ssmall}{3}{100}{\ecl/tmethod/UseRandomInputStream.java}
}

\fr{Template Method: esempio \Cil{BankAccount}}{
    \sizedrangedcode{\ssmall}{3}{100}{\ecl/tmethod/BankAccount.java}
}

\fr{Template Method: esempio con i metodi \Cil{default}}{
    \sizedrangedcode{\ssmall}{5}{100}{\ecllambda/interfaces/SimpleIterator.java}
    \sizedrangedcode{\ssmall}{3}{100}{\ecllambda/interfaces/UseSimpleIterator.java}
}



\frs{5}{Strategy: comportamentale, su oggetti}{
    \bl{Intento/motivazione}{
	Definisce una famiglia di algoritmi, e li rende interscambiabili, ossia usabili in modo trasparente dai loro clienti
    }
    \bl{Esempi}{\iz{
	\item Strategia di disposizione di componenti in una GUI (\cil{LayoutManager})
	\item Strategie di confronto fra due elementi per sorting (\cil{Comparable})
	\item Strategie di \cil{map}, \cil{filter}, etc.. negli \cil{Stream}
    }}
    \bl{Soluzione}{\iz{
	\item Gli algoritmi sono realizzati tramite specializzazioni di una classe/interfaccia base
	\item Ai clienti passo un oggetto (di una specializzazione) della classe base
	\item Se la strategia è funzionale si usano facilmente le lambda (e viceversa)
	\item[$\Rightarrow$] \alert{È probabilmente uno dei pattern più importanti (assieme al Factory Methods)}
    }}
}

\fr{Strategy: UML}{
    \fg{height=0.3\textheight}{img/strategy.jpg}
}

\fr{Strategy: Sorting con comparatori}{
    \sizedrangedcode{\ssmall}{5}{100}{\ecl/strategy/UseComparator.java}
}

\fr{Strategy: Caso del BankAccount}{
    \sizedrangedcode{\ssmall}{3}{100}{\ecl/strategy/BankOperationFees.java}
    \sizedrangedcode{\ssmall}{3}{100}{\ecl/strategy/StandardBankOperationFees.java}
}

\fr{Strategy: Caso del BankAccount}{
    \sizedrangedcode{\ssmall}{3}{100}{\ecl/strategy/BankAccount.java}
}

\frs{12}{Strategy vs Template Method}{
  \bl{In comune}{\iz{
    \item Entrambi li si ottengono dall'esigenza di scorporare da una classe la gestione di una strategia o specializzazione
    \item Entrambi richiedono un behaviour aggiuntivo da realizzare
  }}
  \bl{Differenze}{\iz{
    \item Strategy è più flessibile, perché gli oggetti che rappresentano la specializzazione sono liberi dal dover estendere una certa classe, e quindi sono più facilmente riusabili (p.e. un \cil{Comparator} è usabile con collection diverse)
    \item Template Method si integra con il subtyping, e quindi va usato quando a strategie specializzate devono corrispondere classi specializzate
  }}
  \bl{Altre note}{\iz{
    \item Negli \cil{InputStream} le limitazioni del Template Method sono mitigate dal Decorator
    \item Con le lambda, l'approccio a Strategy diventa più naturale
    \item Valutare di usare il Template Method insieme a Strategy, ossia per definire gerarchie di strategie 
  }}
  
}



\frs{5}{Decorator: strutturale, su oggetti}{
    \bl{Intento/motivazione}{
	Aggiunge ad un oggetto ulteriori responsabilità, dinamicamente, e in modo più flessibile (e componibile) rispetto all'ereditarietà.
    }
    \bl{Esempi}{\iz{
	\item Aggiungere (in modo componibile) la gestione ``buffered'' ad uno stream
	\item Aggiungere una barra di scorrimento ad un pannello
	\item Ottenere uno stream ordered da uno unordered
    }}
    \bl{Soluzione}{\iz{
	\item La classe base viene estesa con una nuova classe che è anche wrapper di un oggetto della classe base
	\item Uno o più metodi potrebbero delegare semplicemente all'oggetto wrappato, altri modificare opportunamente, altri essere aggiuntivi
	\item[$\Rightarrow$] può essere visto come variante dello strategy (in cui la strategia è la realizzazione base del comportamento), e potrebbe includere dei template method
    }}
}

\fr{Decorator: UML}{
    \fg{height=0.4\textheight}{img/decorator.jpg}
}

\frs{10}{Esempio di problema}{
    \bl{Data la seguente interfaccia \cil{Pizza}..}{
      \sizedrangedcode{\ssmall}{3}{100}{\ecl/decorator/problem/Pizza.java}
    }
    \bl{Realizzare le seguenti astrazioni}{\iz{
      \item Margherita (6.50 euro, ingredienti: pomodoro + mozzarella)
      \item Aggiunta Salsiccia (1.50 euro), anche doppia o tripla eccetera
      \item Aggiunta Funghi (1 euro), anche doppia o tripla eccetera
      \item Pizza senza glutine (+10\% costo) 
    }}
    \bl{Forniamo la soluzione col decoratore (tipica del concetto di ``ingrediente'')}{\iz{
      \item Class concreta \cil{Margherita}
      \item Decoratore astratto \cil{IngredientDecorator}
      \item Specializzazioni \cil{Salsiccia}, \cil{Funghi} e \cil{GlutenFree}
    }}
}

\fr{Funzionalità di testing}{
  \sizedrangedcode{\ssmall}{23}{100}{\ecl/decorator/problem/Test.java}
}

\fr{Classe \Cil{Margherita}}{
  \sizedrangedcode{\ssmall}{3}{100}{\ecl/decorator/Margherita.java}
}

\fr{Classe \Cil{IngredientDecorator}}{
  \sizedrangedcode{\ssmall}{3}{100}{\ecl/decorator/IngredientDecorator.java}
}

\fr{Classi per gli ingredienti..}{
  \sizedrangedcode{\ssmall}{3}{100}{\ecl/decorator/Salsiccia.java}
  \sizedrangedcode{\ssmall}{3}{100}{\ecl/decorator/Funghi.java}
  \sizedrangedcode{\ssmall}{3}{100}{\ecl/decorator/GlutFree.java}
}



\frs{5}{Iterator: strutturale, su oggetti}{
    \bl{Intento/motivazione}{
	Fornisce un modo per accedere agli elementi di un aggregato, sequenzialmente, senza esporne la rappresentazione interna.
    }
    \bl{Esempi}{\iz{
	\item Gli iteratori delle collezioni di Java
	\item Un iteratore sui componenti di una GUI
	\item Variante \cil{Spliterator} negli \cil{stream}, e \cil{ListIterator}
    }}
    \bl{Soluzione}{\iz{
	\item La classe iterabile (collezione, stream) di partenza fornisce un metodo per creare un iteratore
	\item L'iteratore ha metodi per accedere sequenzialmente agli elementi dell'iterabile
	\item (Con multiple specializzazioni, se ne nascondono i dettagli interni)
    }}
}

\fr{Iterator: UML}{
    \fg{height=0.4\textheight}{img/iterator.png}
}

\fr{Observer: comportamentale, su oggetti}{
    \bl{Intento/motivazione}{
	Definisce una dipendenza dinamica uno-molti: quando uno cambia, molti vengono notificati/aggiornati
    }
    \bl{Esempi}{\iz{
	\item Un componente grafico potrebbe avere agganciati vari osservatori
	\item Qualunque dispositivo del S.O. potrebbe essere agganciato a vari osservatori
    }}
    \bl{Soluzione}{\iz{
	\item Un \emph{subject} ha un metodo per collegargli un nuovo ascoltatore
	\item Quando una determinata condizione vale nel \emph{subject}, in tutti gli ascoltatori viene chiamato un certo metodo (\cil{notify} o \cil{update})
	\item Molto usato nei sistemi embedded, in OOP spesso solo per le GUI
    }}
}

\fr{Observer: UML}{
    \fg{height=0.5\textheight}{img/observer.jpg}
}

\fr{Observer: Classi generiche per osservazioni}{
    \sizedrangedcode{\scriptsize}{3}{100}{\ecl/observer/ESource.java}
    \sizedrangedcode{\scriptsize}{3}{100}{\ecl/observer/EObserver.java}
}

\fr{Observer: Una collezione emettitrice di eventi}{
    \sizedrangedcode{\scriptsize}{3}{100}{\ecl/observer/SetWithEvents.java}
}

\fr{Observer: Uso della collezione}{
    \sizedrangedcode{\ssmall}{3}{100}{\ecl/observer/UseSetWithEvents.java}
}


\fr{Adapter: strutturale, su classi/oggetti}{
    \bl{Intento/motivazione}{
	Consente ad una classe di adattarsi all'interfaccia (diversa) richiesta da un cliente
    }
    \bl{Esempi}{\iz{
	\item Il metodo \cil{Arrays.asList} (adatta un array ad una lista)
	\item La classe \cil{InputStreamReader} (adatta un \cil{InputStream} ad un \cil{Reader})
    }}
    \bl{Soluzione}{\iz{
	\item Si crea una nuova classe (adapter) che implementa l'interfaccia richiesta e wrappa l'oggetto di partenza (adaptee)
	\item L'adapter redirige opportunamente le chiamata all'adaptee (o via inheritance o via delegazione)
    }}
}

\fr{Adapter: UML}{
    \fg{height=0.4\textheight}{img/adapter2.jpg}
    \bl{Varianti}{\iz{
    \item In Java addirittura l'\cil{Adapter} può essere una inner class (istanza) per \cil{Adaptee}{\iz{
      \item E' considerato uno degli utilizzi più rilevanti per le inner class (istanza)
      \item Consente un incapsulamento ottimale dell'\cil{Adapter}
    }}
    \item \cil{Adapter} potrebbe estendere l'\cil{Adaptee} invece che averne un riferimento
    }}
}

\fr{Esempio: handler d'eventi locale in una GUI}{
  \sizedrangedcode{\ssmall}{9}{100}{\ecl/adapter/GUI.java}
}

\section{Pattern creazionali}

\fr{Pattern creazionali}{
  \bl{Pattern creazionali}{\iz{
    \item Singleton, Factory Method, Abstract Factory, Builder, Prototype
  }}
  \bl{Motivazioni generali}{\iz{
    \item Costruire correttamente oggetti o famiglie di oggetti è una responsabilità non sempre banale, e pertanto può essere meglio scorporarla dalle classi stesse
    \item La costruzione di un oggetto è tipicamente fuori dalla sua interfaccia d'uso, e questo quindi non consente un buon disaccoppiamento e riuso
  }}
      \bl{Una nota}{\iz{
      \item Sebbene esista una differenza tecnica chiara fra questi, è possibile ottenere soluzioni ibride: anche in rete/letteratura non c'è assoluta coerenza nella loro interpretazione
    }}

}

\frs{5}{Factories (ossia ``fabbriche'' di oggetti)}{
  \bl{Static Factory}{\iz{
    \item Una classe ha un metodo statico per generare istanze di sue (o altre) specializzazioni
    \item Es.: l'interfaccia \cil{Stream}, il Singleton è un caso particolare..
  }}
  \bl{Simple Factory}{\iz{
    \item Come Static Factory, ma il metodo è non statico e in una classe separata (factory)
  }}
  \bl{Factory Method}{\iz{
    \item Come Simple Factory, ma la costruzione avviene in sottoclassi della factory, che è un interfaccia
  }}
  \bl{Abstract Factory}{\iz{
    \item Come Factory Method, ma consente di costruire più oggetti tra loro correlati
  }}
}

\fr{Static e Simple Factory}{
  \fg{width=0.8\textheight}{img/fac1.png}
  \fg{height=0.6\textheight}{img/fac2.png}
}

\fr{Static Factory: Esempio \Cil{Persona}}{
    \sizedrangedcode{\ssmall}{3}{100}{\ecl/factory/Persona.java}
}


\fr{Factory Method e Abstract Factory}{
  \fg{width=0.8\textheight}{img/fac3.png}
  \fg{width=0.8\textheight}{img/fac4.png}
}


\fr{Factory Method: creazionale, su oggetti}{
    \bl{Intento/motivazione}{
	Definisce una interfaccia per creare oggetti, lasciando alle sottoclassi il compito di decidere quale classe istanziare e come
    }
    \bl{Esempi}{\iz{
	\item Un framework deve creare oggetti, ma sue specializzazioni devono crearne versioni specializzate
    }}
    \bl{Soluzione}{\iz{
	\item Una interfaccia creatrice fornisce il metodo factory col compito di creare e ritornare l'oggetto
	\item Tale interfaccia viene poi specializzata, e incapsula la logica di creazione dell'oggetto
	\item[$\Rightarrow$] ..spesso frainteso con static o simple factory
    }}
}

\fr{Factory Method: UML}{
    \fg{height=0.45\textheight}{img/factorymethod.png}
}

\fr{Factory Method: Esempio \Cil{Persona} e \Cil{FactoryPersona}}{
    \sizedrangedcode{\ssmall}{3}{100}{\ecl/factory/person/Person.java}
    \sizedrangedcode{\ssmall}{3}{100}{\ecl/factory/person/PersonFactory.java}
}


\fr{Abstract Factory: creazionale, su oggetti}{
    \bl{Intento/motivazione}{
	Definisce una interfaccia per creare famiglie di oggetti tra loro correlati o dipendenti, lasciando alle sottoclassi il compito di decidere quali classe istanziare e come
    }
    \bl{Esempi}{\iz{
	\item Un framework deve creare una famiglia di oggetti, ma sue specializzazioni devono crearne versioni specializzate
    }}
    \bl{Soluzione}{\iz{
	\item Una interfaccia creatrice fornisce i metodi factory col compito di creare e ritornare gli oggetti
	\item Tale interfaccia viene poi specializzata, e incapsula la logica di creazione degli oggetti
    }}
}


\fr{Abstract Factory: An example}{
    \fg{height=0.7\textheight}{\ecl/factory/gui_factory.png}
}


\fr{Abstract Factory: Esempio \Cil{GUIFactory}}{
    \sizedrangedcode{\ssmall}{3}{100}{\ecl/factory/GUIFactory.java}
}

\fr{Abstract Factory: Specializzazione \Cil{MyGUIFactory}}{
    \sizedrangedcode{\ssmall}{6}{100}{\ecl/factory/MyGUIFactory.java}
}

\fr{Abstract Factory: Uso di \Cil{GUIFactory}}{
    \sizedrangedcode{\ssmall}{5}{100}{\ecl/factory/UseGUIFactory.java}
}

\fr{Variante dell'interfaccia con defaults}{
    \sizedrangedcode{\ssmall}{5}{100}{\ecl/factory/alternate/GUIFactory.java}
}

\frs{5}{Builder: creazionale, su oggetti}{
    \bl{Intento/motivazione}{
	Definisce una strategia separata per la creazione step-by-step di un oggetto
    }
    \bl{Esempi}{\iz{
	\item \cil{Stream.Builder}, \cil{StringBuilder}, \cil{StringBuffer}
	\item Una pipeline per gli stream assomiglia ad un builder per un iteratore
    }}
    \bl{Soluzione}{\iz{
	\item Una classe separata ha i metodi per settare le varie proprietà dell'oggetto da costruire, e quando tutto è pronto si chiama un metodo di building
	\item Spesso è comodo usare una interfaccia ``fluent'', ossia dove i setter tornano \cil{this}
	\item Comodo per evitare le proliferazione di costruttori in una classe
	\item Comodo per costruire oggetti immutabili complessi
	\item Dove possibile/utile, lasciare i pochi parametri obbligatori nel costruttore del builder
    }}
}

\fr{Builder: UML}{
    \fg{height=0.45\textheight}{img/builder.png}
}


\fr{Esempio: \Cil{StringBuilder}}{
    \sizedrangedcodet{\ssmall}{6}{100}{\ecl/builder/UseStringBuilder.java}
}

\fr{Esempio: un builder per persone (1/2)}{
    \sizedrangedcodet{\tiny}{3}{20}{\ecl/builder/Person.java}
}

\fr{Esempio: un builder per persone (2/2)}{
    \sizedrangedcodet{\tiny}{21}{100}{\ecl/builder/Person.java}
}

\fr{Esempio: \Cil{UsePersonBuilder}}{
    \sizedrangedcodet{\ssmall}{3}{100}{\ecl/builder/UsePersonBuilder.java}
}

\fr{Altri pattern}{
\bl{Riferimenti}{\iz{
   \item Composite: meccanismo di composizione ad albero di componenti
   \item Memento: per consentire forme di Undo/Redo di modifiche
   \item Lightweight: per gestire un pool di oggetti per ottimizzare performance
   \item Visitor: per incapsulare la logica di visita di un albero
   \item Command: per rappresentare in un argomento il tipo di comando da eseguire
}}
}

\end{document}


