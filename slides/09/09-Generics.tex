\documentclass[presentation]{beamer}
\usepackage{../oop-slides}


\title[OOP09: Generici]{09 \\ Generici \\ (polimorfismo parametrico)}

\begin{document}

\newcommand{\ecl}{\idepath{p09generics}}


\frame[label=coverpage]{\titlepage}

\fr{Outline}{
  \bl{Goal della lezione}{\iz{
  \item Illustrare il problema delle collezioni polimorfiche
  \item Discutere il concetto di polimorfismo parametrico
  \item Illustrare i Generici di Java e alcuni loro vari dettagli
  }}
  \bl{Argomenti}{\iz{
  \item Collezioni con polimorfismo inclusivo
  \item Classi generiche
  \item Interfacce generiche
  \item Metodi generici
  }}
}

\section{Collections con polimorfismo inclusivo}

\fr{Forme di riuso nella programmazione OO}{
  \bl{Composizione (e come caso particolare, delegazione)}{
    Un oggetto è ottenuto per composizione di oggetti di altre classi
  }
  \bl{Estensione}{
    Una nuova classe è ottenuta riusando il codice di una classe pre-esistente
  }
  \bl{Polimorfismo inclusivo (subtyping)}{
    Una funzionalità realizzata per lavorare su valori/oggetti del tipo \cil{A}, può lavorare con qualunque valore/oggetto del sottotipo \cil{B} (p.e., se \cil{B} estende la classe \cil{A}, o se \cil{B} implementa l'interfaccia \cil{A})
  }
  \bl{Polimorfismo parametrico (Java/C\# generics, C++ templates,..)}{
    Una funzionalità (classe o metodo) generica è costruita in modo tale da lavorare uniformemente su valori/oggetti indipendentemente dal loro tipo: tale tipo diventa quindi una sorta di parametro addizionale
  }
}

\fr{Astrazioni uniformi con le classi}{
  \bl{Astrazioni uniformi per problemi ricorrenti}{\iz{
    \item Durante lo sviluppo di vari sistemi si incontrano problemi ricorrenti che possono trovare una soluzione comune
    \item In alcuni casi queste soluzioni sono fattorizzabili (per astrazione) in una singola classe altamente riusabile
    \item Esempi: collezioni, stream, funzioni, produttori, consumatori, predicati
  }}
  \bl{Un caso fondamentale: le \alert{collection}}{\iz{
    \item Una collection è un oggetto il cui compito è quello di immagazzinare il riferimento ad un numero (tipicamente non precisato) di altri oggetti 
    \item Fra i suoi compiti c'è quello di consentire modifiche ed accessi veloci all'insieme di elementi di tale collezioni
    \item Varie strategie possono essere utilizzate, seguendo la teoria/pratica degli algoritmi e delle strutture dati
  }}
}

\fr{Un esempio: \Cil{IntVector}}{
  \bl{Collection \Cil{IntVector}}{\iz{
    \item Contiene serie numeriche (vettori) di dimensione non nota a priori
    \item Ossia, a lunghezza variabile..
  }}
  \fg{height=0.6\textheight}{img/uml-int-vector-abstracted.pdf}
}

\fr{\Cil{UseIntVector}}{
  \sizedrangedcode{\scriptsize}{3}{100}{\ecl/abstractions/UseIntVector.java}
}

\fr{\Cil{IntVector} -- implementazione}{
  \bl{Collection \Cil{IntVector}}{\iz{
    \item Contiene serie numeriche (vettori) di dimensione non nota a priori
    \item Realizzata componendo un array che viene espanso all'occorrenza
  }}
  \fg{height=0.6\textheight}{img/uml-int-vector.pdf}
}


\fr{\Cil{IntVector} pt 1}{
  \sizedrangedcode{\scriptsize}{3}{25}{\ecl/abstractions/IntVector.java}
}

\fr{\Cil{IntVector} pt 2}{
  \sizedrangedcode{\scriptsize}{26}{50}{\ecl/abstractions/IntVector.java}
}

\fr{Un primo passo verso l'uniformità}{
  \bl{Solo elenchi di \cil{int}?}{\iz{
    \item L'esperienza porterebbe subito alla necessità di progettare vettori di \cil{float}, \cil{double}, \cil{boolean},.. ossia di ogni tipo primitivo
    \item E poi, anche vettori di \cil{String}, \cil{Date}, eccetera
    \item L'implementazione sarebbe analoga, ma senza possibilità di riuso..
  }}
  \bl{Collection uniformi ``monomorfiche''}{\iz{
    \item Una prima soluzione del problema la si ottiene sfruttando il polimorfismo inclusivo e la filosofia ``everything is an object'' (incluso l'uso dell'autoboxing dei tipi primitivi)
    \item Si realizza unicamente un \cil{ObjectVector}, semplicemente sostituendo \cil{int} con \cil{Object}
    \item Si inserisce qualunque elemento (via upcast implicito)
    \item Quando si riottiene un valore serve un downcast esplicito
  }}
}

\fr{Da \Cil{IntVector} a \Cil{ObjectVector}}{
  \fg{height=0.6\textheight}{img/uml-obj-vector.pdf}
}

\fr{\Cil{UseObjectVector}}{
  \sizedrangedcode{\ssmall}{3}{100}{\ecl/abstractions/UseObjectVector.java}
}

\fr{Un altro caso di collection, la list linkata \Cil{ObjectList}}{
  \sizedrangedcode{\ssmall}{3}{100}{\ecl/abstractions/ObjectList.java}
}

\fr{\Cil{UseObjectList}}{
  \sizedrangedcode{\scriptsize}{3}{100}{\ecl/abstractions/UseObjectList.java}
}


\fr{La necessità di un approccio a polimorfismo parametrico}{
  \bl{Prima di Java 5}{\iz{
    \item Questo era l'approccio standard alla costruzione di collection
    \item Java Collection Framework --- una libreria fondamentale
  }}
  \bl{Problema}{\iz{
    \item Con questo approccio, nel codice Java risultavano molti usi di oggetti simili a \cil{ObjectVector} o \cil{ObjectList}
    \item Si perdeva molto facilmente traccia di quale fosse il contenuto..{\iz{
      \item contenevano oggetti vari? solo degli \cil{Integer}? solo delle \cil{String}?
    }}
    \item Il codice conteneva spesso dei downcast sbagliati, e quindi molte applicazioni Java fallivano generando dei \cil{ClassCastException}
  }}
  \bl{Più in generale}{
    Il problema si manifesta ogni volta che voglio collezionare oggetti il cui tipo non è noto a priori, ma potrebbe essere soggetto a polimorfismo inclusivo
  }
}

%\fr{Il problema con le composizioni multiple}{
%  \sizedcode{\ssmall}{code/LampsRow.java}
%}


\fr{Polimorfismo parametrico}{
  \bl{Idea di base}{\iz{
    \item Dato un frammento di codice \cil{F} che lavora su un certo tipo, diciamo \cil{String}, se potrebbe anche lavorare in modo uniforme con altri\dots
    \item \dots lo si rende parametrico, sostituendo a \cil{String} una sorta di variabile \cil{X} (chiamata \alert{type-variable}, ossia una variabile che contiene un tipo)
    \item A questo punto, quando serve il frammento di codice istanziato sulle stringhe, si usa \cil{F<String>}, ossia si richiede che \cil{X} diventi \cil{String}
    \item Quando serve il frammento di codice istanziato sugli integer, si usa \cil{F<Integer>}
  }}
  \bl{Java Generics}{\iz{
    \item Classi/interfacce/metodi generici
    \item Nessun impatto a run-time, per via dell'implementazione a ``erasure''{\iz{
      \item \cil{javac} ``compila via i generici'', quindi la JVM non li vede
    }}
  }}
}

\section{Classi generiche}

\fr{La classe generica \Cil{List}}{
  \sizedcode{\scriptsize}{code/List.java}
}

\fr{Uso di una classe generica}{
  \sizedrangedcode{\scriptsize}{3}{100}{\ecl/generics/UseList.java}
}

\fr{Terminologia, e elementi essenziali}{
  \bl{Data una classe generica \cil{C<X,Y>}..}{\iz{
    \item \cil{X} e \cil{Y} sono dette le sue \alert{type-variable}
    \item \cil{X} e \cil{Y} possono essere usati come un qualunque tipo dentro la classe (con alcune limitazioni che vedremo)
  }}
  \bl{I clienti delle classi generiche}{\iz{
    \item Devono usare \alert{tipi generici}: versioni ``istanziate'' delle classi generiche\iz{
      \item \cil{C<String,Integer>}, \cil{C<C<Object,Object>,Object>}
      \item Non \cil{C} senza parametri, altrimenti vengono segnalati dei warning
    }
    \item Ogni type-variable va sostituita con un tipo effettivo, ossia con un \alert{parametro}, che può essere{\iz{
      \item una classe (non-generica), p.e. \cil{Object}, \cil{String},..
      \item una type-variable definita, p.e. \cil{X,Y} (usate dentro la classe \cil{C<X,Y>})
      \item un tipo generico completamente istanziato, p.e. \cil{C<Object,Object>}
      \item ..o  parzialmente istanziato, p.e. \cil{C<Object,X>} (in \cil{C<X,Y>})
      \item NON con un tipo primitivo
    }}
  }}
}


\fr{La classe generica \Cil{Vector}}{
  \sizedcode{\footnotesize}{code/Vector.java}
}

\fr{Uso di \Cil{Vector<X>}}{
  \sizedrangedcode{\scriptsize}{3}{100}{\ecl/generics/UseVector.java}
}

\fr{Implementazione di \Cil{Vector} pt 1}{
  \sizedcode{\scriptsize}{code/VectorA.java}
}

\fr{Implementazione di \Cil{Vector} pt 2}{
  \sizedcode{\scriptsize}{code/VectorB.java}
}



\fr{La classe generica \Cil{Pair<X,Y>}}{
  \srcode{\scriptsize}{3}{100}{\ecl/generics/Pair.java}
}

\fr{Uso di \Cil{Pair<X,Y>}}{
  \srcode{\ssmall}{3}{100}{\ecl/generics/UsePair.java}
}

\frs{10}{Inferenza dei parametri}{
  \bl{Un problema sintattico dei generici}{\iz{
    \item Tendono a rendere il codice più pesante (``verbose'')
    \item Obbligano a scrivere i parametri anche dove ovvi, con ripetizioni
  }}
  \bl{L'algoritmo di type-inference nel compilatore}{\iz{
    \item Nella \cil{new} si possono tentare di omettere i parametri (istanziazione delle type-variable), indicando il ``diamond symbol'' \cil{<>}
    \item Il compilatore cerca di capire quali siano questi parametri guardando gli argomenti della \cil{new} e l'eventuale contesto dentro il quale la \cil{new} è posizionata, per esempio, se assegnata ad una variabile
    \item Nel raro caso in cui non ci riuscisse, segnalerebbe un errore a tempo di compilazione.. quindi tanto vale provare!
    \item Ricordarsi \cil{<>}, altrimenti viene confuso con un \alert{raw type}, un meccanismo usato per gestire il legacy con le versioni precedenti di Java
    }}
  \bl{La local variable type inference}{\iz{
    \item in genere è alternativa al simbolo \cil{<>}
  }}
  
}

\fr{Esempi di inferenza}{
  \srcode{\ssmall}{3}{100}{\ecl/generics/UsePair2.java}
}

\fr{I vantaggi dei generici}{
  \bx{Coi generici, Java diventa un linguaggio molto più espressivo!}
  \bl{Svantaggi}{\iz{
    \item Il linguaggio risulta più sofisticato, e quindi complesso
    \item Se non ben usati, possono minare la comprensibilità del software
    \item Non vanno abusati!!
    \item Gli aspetti più avanzati dei generici, che vedremo, sono considerati troppo complessi
  }}
  \bl{Vantaggi -- se ben usati}{\iz{
    \item Codice più comprensibile
    \item Maggiori possibilità di riuso di funzionalità (quasi d'obbligo oramai)
    \item Codice più sicuro (safe) -- il compilatore segnala errori difficili da trovare altrimenti
  }}
}

\section{Interfacce generiche}

\fr{Interfacce generiche}{
  \bl{Cosa è una interfaccia generica}{\iz{
    \item \`E una interfaccia che dichiara type-variables: \cil{interface I<X,Y>\{..\}}
    \item Le type-variable compaiono nei metodi definiti dall'interfaccia
    \item Quando una classe la implementa deve istanziare le type variabile (o assegnarle ad altre type-variable se essa stessa è generica)
  }}
  \bl{Utilizzi}{
    Per creare contratti uniformi che non devono dipendere dai tipi utilizzati
  }
  \bl{Un esempio notevole, gli \alert{Iteratori}}{\iz{
    \item Un iteratore è un oggetto usato per accedere ad una sequenza di elementi
    \item Ne vedremo ora una versione semplificata -- diversa da quella delle librerie Java
  }}
}

\fr{L'interfaccia \Cil{Iterator}}{
    \srcode{\normalsize}{3}{100}{\ecl/iterators/Iterator.java}
}

\fr{Implementazione 1: \Cil{IntRangeIterator}}{
  \srcode{\scriptsize}{3}{100}{\ecl/iterators/IntRangeIterator.java}
}

\fr{Implementazione 2: \Cil{ListIterator}}{
  \srcode{\scriptsize}{3}{100}{\ecl/iterators/ListIterator.java}
}

\fr{Implementazione 3: \Cil{VectorIterator}}{
  \srcode{\scriptsize}{3}{100}{\ecl/iterators/VectorIterator.java}
}

\fr{\Cil{UseIterators}: nota l'accesso uniforme!}{
  \srcode{\ssmall}{3}{100}{\ecl/iterators/UseIterators.java}
}

\section{Metodi generici}

\fr{Metodi generici}{
  \bl{Metodo generico}{
    Un metodo che lavora su qualche argomento e/o valore di ritorno in modo independente dal suo tipo effettivo. Tale tipo viene quindi astratto in una type-variable del metodo.
  }
  \bl{Sintassi}{\iz{
    \item def: \cil{<X1,..,Xn> ret-type nome-metodo(formal-args)\{...\}}
    \item call: \cil{receiver.<X1,..,Xn>nome-metodo(actual-args)\{...\}}
    \item call con inferenza, stessa sintassi delle call standard, ossia senza \cil{<>}
  }}
  \bl{Due casi principali, con medesima gestione}{\iz{
    \item Metodo generico (statico o non-statico) in una classe non generica
    \item Metodo generico (non-statico) in una classe generica
    \item[$\Rightarrow$] Il primo dei due molto più comune..
  }}
}

\fr{\Cil{UseIterators}: Definizione e uso di metodo generico}{
  \srcode{\ssmall}{3}{100}{\ecl/iterators/UseIterators2.java}
}

\fr{Esempio di metodo generico in classe generica}{
  \sizedcode{\ssmall}{code/UseGenMeth.java}
}

\fr{Notazione UML: non del tutto standard}{
  \fg{height=0.8\textheight}{img/uml-generics.pdf}
}

\section{Java Wildcards}

\fr{Java Wildcards}{
  \bl{Osservazione}{\iz{
    \item Esistono situazioni in cui un metodo debba accettare come argomento non solo oggetti di un tipo \cil{C<T>}, ma di ogni \cil{C<S>} dove \cil{S <: T}
    \item Esempio: un metodo statico \cil{printAll()} che prende in ingresso un iteratore e ne stampa gli elementi
    \item \`E realizzabile con un metodo generico, ma ci sono casi in cui si vorrebbe esprimere un tipo che raccolga istanziazioni diverse di una classe generica
  }}
  \bl{Java Wildcards}{\iz{
    \item Un meccanismo avanzato, quello inventato più di recente (2004-2006){\iz{
      \item {\scriptsize (Igarashi \& Viroli) + (Bracha \& Gafter) + (Torgersen \& Hansen \& von der Ahé)}
    }}
    \item Fornisce dei nuovi tipi, chiamati Wildcards 
    \item Simili a interfacce (non generano oggetti, descrivono solo contratti)
    \item Generalmente usati come tipo dell'argomento di metodi
  }}
}


\fr{Java Wildcards}{
  \sizedcode{\footnotesize}{code/Numbers.java}
  \sizedcode{\footnotesize}{code/Wildcards.java}
}

\fr{Java Wildcards}{
  \bl{3 tipi di wildcard}{\iz{
    \item Bounded (covariante): \cil{C<? extends T>}{\iz{
      \item accetta un qualunque \cil{C<S>} con \cil{S <: T}
    }}
    \item Bounded (controvariante): \cil{C<? super T>}{\iz{
      \item accetta un qualunque \cil{C<S>} con \cil{S >: T}
    }}
    \item Unbounded: \cil{C<?>}{\iz{
      \item accetta un qualunque \cil{C<S>}
    }}    
  }}
  \bl{Uso delle librerie che dichiarano tipi wildcard}{\iz{
    \item Piuttosto semplice, basta passare un argomento compatibile o si ha un errore a tempo di compilazione
  }}
  \bl{Progettazione di librerie che usano tipi wildcard}{\iz{
    \item Molto avanzato: le wildcard pongono limiti alle operazioni che uno può eseguire, derivanti dal principio di sostituibilità
  }}
}

\fr{Esempio Wildcard}{
  \sizedrangedcode{\scriptsize}{5}{100}{\ecl/wildcard/Wildcard.java}
}






\end{document}


