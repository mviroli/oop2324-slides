\documentclass[presentation]{beamer}
\usepackage{../oop-slides}

\title[OOP00 -- Intro]{00 \\ Programmazione ad Oggetti \\ Introduzione al corso}

\begin{document}

\frame[label=coverpage]{\titlepage}

\fr{Docenti}{
\centering
  \bl{Titolare del corso: Prof. Mirko Viroli}{
    \iz{ 
      \item e-mail | \texttt{mirko.viroli@unibo.it}
      \item homepage | \texttt{https://www.unibo.it/sitoweb/mirko.viroli}
    }
  }
  \bl{Modulo di laboratorio: Prof. Danilo Pianini}{
    \iz{ 
      \item e-mail | \texttt{danilo.pianini@unibo.it}
      \item homepage | \texttt{https://www.unibo.it/sitoweb/danilo.pianini}
    }
  }
  \bl{\dots e tutors in laboratorio}{\dots}
}

\fr{Contatti con gli studenti}{
  \bl{Chi contattare}{\iz{
  \item Mirko Viroli: contenuti tecnici del corso + organizzazione del corso
  \item Pianini: strumenti e esercizi di laboratorio
  }}
  \bl{Attraverso il forum}{\iz{ 
  \item Per domande la cui risposta è di interesse generale \\ (e quindi tutte le domande tecniche)
  }}
  \bl{Via mail al docente}{\iz{ 
  \item Per questioni personali
  \item Per domande che consentono risposte concise
  }}
  \bl{Ricevimento}{\iz{ 
  \item (annunciato nelle Home Page dei docenti)
}}}

\fr{Sito del corso}{
  \bl{Sito \alert{virtuale} di Ateneo}{\iz{
  \item \myurl{https://virtuale.unibo.it/course/view.php?id=48035}
  \item sarà il luogo degli avvisi (e notifiche), forum di discussione, produzione di materiale
  \item tutti gli studenti che seguono il corso si iscrivano, e lo tengano d'occhio
  \item in particolare: avendo sempre le slide sottomano
  }}
}


\fr{Organizzazione generale del corso}{
  \bl{Lezioni aula (due da 3 ore la settimana)}{\iz{ 
  \item Illustrano i concetti teorici, metodologici e pratici
  \item Basate su slide proiettate (ma non solo)
  }}
  \bl{Laboratorio (turni da 3-4 ore a settimana)}{\iz{
  \item A giorni alterni (il che vi lascia un giorno libero)
  \item Illustra ulteriori aspetti metodologici e pratici
  \item Con esercizi necessari alla comprensione e alla sperimentazione
  \item \alert{\`E parte integrante del corso}
  }}
  \bl{Studio a casa (almeno 4 ore a settimana) -- p.e., nel giorno libero}{\iz{
  \item Rilettura slide, esperimenti pre- e/o post-laboratorio
  \item È praticamente necessario se volete rimanere in pari\dots
}}}

\fr{Programma (di massima) del corso}{
  \bl{Parti principali}{\iz{ 
  \item Elementi base di programmazione OO e Java
  \item Polimorfismo (ereditarietà, subtyping, genericità)
  \item Librerie (I/O, grafica, concorrenza)
  \item Integrazione col paradigma funzionale (lambda, streams)
  \item Pattern e buone pratiche di programmazione
  %\item Elementi di programmazione C\#
}}}

\fr{Testi di riferimento (non necessario l'acquisto)}{
  \bl{Programmazione in Java}{\iz{ 
  \item B.Eckel. Thinking in Java, 4th edition.
  \item J.Block. Effective Java, 3rd edition.
  \item R.Warburton. Java 8 Lambdas.
  }}
  %\bl{Programmazione in C\#}{\iz{ 
  %\item Jon Skeet. C\# in depth, 3rd edition.
  %}}
  \bl{Altri riferimenti}{\iz{
  \item E.Gamma et.al. Design Patterns Elements of Reusable Object-Oriented Software.
  \item R.Martin. Clean Code: A Handbook of Agile Software Craftsmanship
  %\item Java e C\# online documentation (tutorials, Language Specification, APIs)
} }}

\frs{15}{Software}{
  \bl{Java}{\iz{ 
  \item Framework Java: OpenJDK 17 (Open Java Development KIT){\iz{
    \item \myurl{https://adoptium.net/}
    }}
  \item Integrated Development Environment: Visual Studio Code + Java plugin{\iz{
    \item \myurl{https://code.visualstudio.com/}
    \item \myurl{https://code.visualstudio.com/docs/languages/java}
    }}
  \item Altri strumenti{\iz{
      \item Git
      \item Gradle
  }}}}
  %\bl{Altri framework -- parte finale del corso}{\iz{ 
  %\item Visual Studio .NET (Microsoft), oppure
  %\item framework Mono e IDE Monodevelop/Rider (Linux e MacOs)
  %}}  
  \bl{Istruzioni sull'installazione (sul PC di casa)}{\iz{ 
  \item Già disponibili su ``virtuale''
  \item \myurl{https://unibo-oop.github.io/software-installation/}
  \item Molto importante rendersi operativi a casa nel giro di una settimana!
  \item L'uso di VSCode sarà necessario solo fra 2/3 settimane
  \item[$\Rightarrow$] sarebbe consigliato l'uso del sistema operativo Linux
  }}
}

\fr{Sul ruolo di questo corso, a.k.a. ``Programmazione 2''}{
    \bl{Elementi essenziali}{\iz{
    \item Costruzione del software, e quindi di sistemi
    \item Analisi problemi, e organizzazione di soluzioni
    \item Tecniche base ed (alcune) avanzate di programmazione ad oggetti
    \item Introduzione a principi/tecniche di ``programmazione moderna''
    \item Elementi di gestione di un progetto software
    \item Utilizzo di strumenti integrati di sviluppo
    }}
    \bl{Questo corso nel più ampio contesto dei vostri studi}{\iz{
    \item Competenze a curriculum
    \item Enfasi sull'approccio metodologico
    \item Target di qualità piuttosto elevato
    \item È cruciale dedicargli subito il tempo necessario
    }}
}

\frs{5}{Esame}{
  \bl{Prova scritta}{\iz{ 
  \item Durata 1.5 ore circa, svolta in laboratorio
  \item Verifica mirata di capacità tecniche, di problem-solving, buona progettazione OO
  \item[$\Rightarrow$] Forniremo esercizi-tipo in laboratorio
  \item[$\Rightarrow$] I temi d'esame del passato sono disponibili e abbastanza indicativi
  \item[$\Rightarrow$] Svolta in laboratorio
  }}
  \bl{Discussione progetto}{\iz{
  \item Progetto sviluppato in gruppo, 70 ore circa a testa (vedi i 12 CFU)
  \item Concordato col docente prima di iniziare
  \item Da relazionare con qualità, poi discusso su appuntamento
  \item Consegne con deadlina scelta da voi (entro 4 mesi) ma a quel punto stringente: \\ non durante le lezioni, idealmente entro il 18/2/2024
  \item[$\Rightarrow$] I dettagli discussi a metà corso
  \item[$\Rightarrow$] Regole d'esame (in versione draft) su ``virtuale'' molto presto

}}
}

\fr{Prerequisiti}{
    \bl{Buona conoscenza}{\iz{
    \item tecniche di programmazione imperativa/strutturata
    \item costruzione e comprensione di semplici algoritmi e strutture dati
    }}
    \bl{Attenzione a chi è già ``fluente'' in linguaggi ad oggetti}{\iz{
    \item Java o C\#
    \item disimparare le ``bad practice'' richiede umiltà e fatica...
    }}
}



\end{document}
